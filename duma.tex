% Beginning of the end

\NewChapter{Executive summary}

\NewSection{Purpose} % <<<

\Topic{What}
\Menugene is a system of software for computer aided dietary counselling.
Its ultimate goal is to help ordinary people in the assembly of daily
and weekly dietary menus---what to have for breakfast, lunch etc. so that
their nutritional needs, dietary allowances and taste are all satisfied.

% <[consumer objectives]>
% [SKIP energy, carbohydrate, lipids, protein, sugar, vitamins]
\Topic<consumer objectives>
From the system's point of view a person's \Subject{nutritional needs}
are defined by the optimal daily consumption of elementary ingredients,
such as carbohydrate, protein, and lipids.  \Subject{Dietary allowances}
restrict the range of dishes consumable by the subject, as in the case of
people with milk susceptibility.  \Subject{Taste} expresses preferences
towards or against particular kinds of meals, or any combinations thereof.

% {intended audience}
The \Keyword<intended audience> extends from the ill to healthcare
providers.  \Subject{People with diabetic disabilities} benefit in
breaking the routine meals by discovering new recipes that are as fitting
to their special needs as the ones prescribed by dietary consultants.
\Subject{Healthy individuals} conscious about their condition will find
it a cheap and easy way to improve their nutrition.  \Subject{Domestic
partners} are relieved of deciding what to prepare for their families,
so that everyone's taste is respected.  \Subject{Diatery experts} may be
able to give their clients better advice.  Finally, \Subject{healthcare
providers} may find it appealing as a low-cost addition to their existing
portfolio.

% {use cases}
\Menugene has three \Keyword[levels of application]{use cases}.
The mid-level usage enables consumers to store, represent and communicate
diatery information \emph{structurally}: recipes, nutritional needs,
dietary allowances, taste.  The high-level use case implies the previous
one, and in addition the system constructs (\Term{generates}) a menu
plan by artificial intelligence methods, attempting to observe all the
constraints expressed with regards to an ideal menu.  On the lowest level
\Menugene is \emph{designed} to be a platform for research in these fields
of information and knowledge representation, and artificial intelligence.

%
% figs/sysinf.tex
%

\begin{NewFig}{sysinf}
\IncludeFig{sysinf}
\caption{Information flow and levels of application of \Menugene}
\end{NewFig}

% End of figs/sysinf.tex


% {deployment configurations}
\Menugene can be deployed and accessed \Keyword[several ways]{deployment
configurations}, as illustrated in \Figure{sysact}.  In the simplest
case (\Figure*{sysact-1}) the consumer is in direct interaction with
the system, telling about \HisHer nutritional needs (perhaps deriving
them from popular literature), dietary allowances, and preferences, then
interpreting the auto-generated menu plan without external assistance.
This setting is well-suited for voluntary \AndOr regular uses of the
service, when the fitting of the menu is not critical or the consumer
has already gathered enough experience with it.  It is also possible
to draw on a third-party consultant (\Figure*{sysact-2}) estimating the
personal requirements and formulating professional advice based on the
machine-generated plans.  This case the consumer may not even be aware
of the working of the computer aid, which is convenient for those who
do not wish to engage in technical details.  The configuration depicted
in \Figure{sysact-3} is a synthesis of the former ones: consumers are
given freedom in specify their preferences, while the responsibility of
determining diatery requirements remains on the shoulders of experts.
Finally (\Figure*{sysact-4}), the system can be envisaged as an integrated
part of another comprehensive advisory system, such as \Link\Prog{Cordelia}.
This setting is best for healthcare providers whose existing services can
gain from the structured personal dietary information held by \Menugene.

%
% figs/sysact.tex
%

\begin{NewFig}{sysact}
\FigPair{sysact}%
	[1]{Individual usage}%
	[2]{Proxied access}
\FigPair{sysact}%
	[3]{Assisted usage}%
	[4]{Integrated deployment}
\caption{Example deployment configurations of \Menugene}
\end{NewFig}

% End of figs/sysact.tex


The presence of `developers' in the figures is meant to emphasize
the importance of the continuous maintenance and enhancement of 
the \Menugene databases, without which the system would quickly
degrade into obsoleteness.
% Purpose >>>

\NewSection{Structure} % <<<

% {layers}
% [SKIP ears, mouth, stomach]
The system consists of several components which we can categorize
roughly into three groups, \Keyword[layered]{layers} upon the top of
one another.  (\FIGure{sysarch-1}) The first layer interacts with
the \Subject{end-user}, accepting requests and presenting results
in human-friendly fashion.  The decoded queries are passed onto the
component that has a global view on the problem to be solved and
\Subject{controls} its resolution, which is done at the bottom by
specialized, reusable \Subject{libraries}.  The communication within
the system and toward end-users is mostly networked, which allows
more scalability and flexibility in deployment.  \FIGure{sysarch-2}
shows the \emph{use}-relationships between the actors, whose detailed
description follows.

%
% figs/sysarch.tex
%

\begin{NewFig}{sysarch}
\FigPair{sysarch}%
	[1]{Layers of \Menugene}\vrule
	[2]{Current structure of \Menugene.  Underlined labels mean
	the component is not an in-house development.  The dashing of
	\Tech{MenuGeneAdmin} is meant to symbolize the author did not
	work on it.}
\FigPair{sysarch}%
	[3]{A possible architecture for the enterprise market.}
\caption[System structure]%
	{System structure.  Arrows express \emph{use}-relationship and
	direction.  The line style indicates the kind of communication
	transport.  Solid lines: local, dotted lines: (inter)networked,
	dot-dash: \Either[RMI][CORBA].  Nodes with rounded corners are
	visible to end-users.  Box shading emphasizes the centrality of
	a component.}
\end{NewFig}

% End of figs/sysarch.tex


% {[junkie]}
\Subject{Consumers}, who need dietary advice are served by \Keyword%
\Target\Tech{junkie}, the web front-end of \Menugene.  Users can
enter their nutritional needs into HTML forms, the program relays this
information to the controller, and displays the tabulated daily or weekly
menu recommendation along with goodness estimations when the response is
ready.  The interface is multi-lingual (currently English and Hungarian
are supported) and can be embedded into other web services, making it
possible to take user requirements from sources other than its own forms.
\FIGure{junkie} shows screenshots of the interface.

% {[MenuGeneAdmin]}, {[database]}
The system's another user-accessible entry point is \Keyword\Target%
\Tech{MenuGeneAdmin}, which is intended for \Subject{dietary experts}.
This application runs on the local computer and allows editing the
\Menugene database remotely through a graphical user interface.%
%%%%%%%%%
\footnote{The author claims no affiliation with the creation of this
software, and takes no responsibility for it.}
%%%%%%%%%
The \Keyword\Target<database> supply the controller with bulk, static
information essential for the menu generation.  It is static in the sense
that its contents are not subject to change in the course of problem
solving.  The database is loaded with recipes (acquired commercially from
\Prog{QuadroByte}) that the final menu recommendation will be consist of,
groupings of the recipes (such as puddings, soft drinks, or fast food),
and a set of constraints applicable in typical circumstances (i.e. to
people suffering from some common disease).  Privileged dietary experts
have means to improve and customize this data set by \Tech{MenuGeneAdmin}
in the interest of their clientele.

% {rulez}
The last user-visible application of the system is \Keyword\Link\Tech{rulez},
which can be used by all end-users to edit \Link<rules>.  The integration
of this program is still in the alpha stage.

% {[daemon]}
The heart of \Menugene is the \Keyword\Target\Tech{daemon}, which formulates
the problem in exact terms based on the information coming from \Link\Tech%
{junkie} and the static data fetched from the database.  (``\Talk{Given~\dots\
find~\dots\ such that~\dots.}'')  Then it configures the engine and calls upon
for problem resolution.  Depending on the configuration this can be a rather
time-consuming process.  Whilst it is proceeding run-time status information
about the engine is available from the (kind of barefoot) \Link[monitoring
facility]{logging} of the \Tech{daemon}.  Besides the regular network
interface used by \Link\Tech{junkie} the program can be operated from
the command line, and also is accessible via \Cite{CORBA} (not merged
into mainline).  The former is reserved for developers, the latter is
held back until future experimentation.

% {engine}, [MGLib]
Once the \Keyword<engine> starts it works unsupervised until a satisfactory
solution is found.  Menus are generated by \Link[genetic algorithms]{works}
(GAs) in collaboration between two software libraries.  \Prog{GALib} is an
external library (with some local adjustments) implementing generic GAs,
which call back into \Target\Tech{MGLib} for problem-specific algorithmic
bits.  The work of both \Tech{MGLib} and \Link\Tech{daemon} is assisted by
\Link\Tech{GSLib}, which essentially defines a convenient general framework
for \emph{clean} software development.  None of the libraries has any
knowledge about the database and its internal schemata.

% {BITS}
\Keyword\Link\Tech{BITS} of software play an important role in the life of
\Menugene developers.  It is a collection of \Link[miscellany utilities]%
{homegrown utilities} which take no part in the normal operation of the
system, but are supportive of the development process.  They include debug
aids, visualizers, database maintenance tools and more.

% {[future]}
\FIGure{sysarch-3} shows a \Keyword\Target[substantially different]%
{future} system architecture envisaged for the enterprise in admission
that that market demands more scalability than \Menugene offers currently.
In the proposition \Link\Tech{daemon} should lose its central position,
taken over by a \Prog{JBoss} conglomerate in its place.  User-visible
interfaces would be sited in the application server, giving users the
feel of a local-running program by means of dummy proxying mock-ups
(not included in the figure).  Bulk data would be hidden behind a
`knowledge base', which can represent complex relationships between
pieces of information that is hard to model in relational databases.
With respect to the database, the basic data set of \Prog{QuadroByte}
is going to be \Link[superseded]{database migration} by the one of the
U.S. Department of Agriculture (\Cite{USDA}), which is believed to be of
higher quality.  Also, we are planning on making room for dynamic hints
derived from past runs for the \Link\Tech{daemon}, so it could start
off searching for optimal menu plans from points proven to be favorable.
This is called \Term{case-based menu generation}.  All these areas have
been worked on, reaching varying status of readiness.
% Structure >>>

\NewSection[The works]{works} % <<<

In this section we will give a high-level outline of the workings of
\Menugene that produces menus satisfying \Link[requirements]{consumer
objectives}.  In this text we will only cover the underlying theory to
the extent, and from the perspective of that is necessary to facilitate
the understanding of the topic in focus.  Interested readers should follow
up \cite{TDK}.

\Topic{Tale}
The engine operates in the \Term{problem space}.  This imaginary space
is filled with \Term{solutions} to the given problem.  In the context of
\Menugene a solution is, certainly, an optimal weekly menu plan, while
the problem space is the (finite) set of all possible weekly diatery menu
plans.  Thus, the duty of the engine is to \emph{pick out} a solution
from the problem space that matches the consumer's needs as closely as
possible.

% {entities}
The engine \Keyword[utilizes]{entities} \Term{genetic algorithms} (GA)
for this purpose.  GAs maintain a subset of the problem space called
\Term{population}, which restricts what solutions the GA can consider
\emph{at a given time}.  The size of the population is chosen so that
computations over the subset taking linear or higher order time become
feasible.  Solutions elected into the population are the \Term{genomes}.
Which genomes these are is of cardinal importance, since the final
solution to the problem will be chosen from them when the algorithm ends.

% [TODO We don't \Figure{} it; not that I feel very compelled to.]
% [TODO Maybe this figure shouldn't be included at all?]
%
% figs/ga.tex
%

\begin{NewFig}{ga}
\IncludeFig[0][0][framed]{ga}
\caption[Entities involved with the genetic algorithm]%
	{Entities involved with the genetic algorithm.
	Dotted lines link labels to their corresponding drawing.}
\end{NewFig}

% End of figs/ga.tex


% {genetic algorithm}
It follows that the \Keyword[task of the GA]{genetic algorithm} is to
bring forth a population that features the genome which is a (near)
optimal solution to the problem the \Menugene engine set out to solve.
For this the genetic algorithm improves its population incrementally.
Iterations of the GA are called \Term{generations}.  In each generation
the GA \Term{evolves} the current population, changing the constituting
genomes in some or other way around.  In formal terminology some of the
genomes are replaced by solutions from the problem space.  This process
may be driven by the intention to get closer to a set of genomes that
appear to have the desired characteristics, or conversely to back off
because the current population seems to be a dead end.%
\footnote{The special case is when the evolution does not have any
particular goal in mind; this is called \Term{random search}.}

Actually, the population is evolved by considering small subsets thereof.
This is why a final \Term{selection} is done before the next generation,
taking all proposed members of the new population into account at
the same time.  The algorithm ends when some \Term{terminal condition}
is reached.  The following pseudo-code summarizes the main points:

\begin{lstlisting}[%
	keywords={genetic_algorithm,evolution,selection,%
		terminal_condition,best_of},%
	emph={population}]
sub genetic_algorithm(population)
{
	population = selection(evolution(population))
		until terminal_condition(population);
	return best_of(population);
}
\end{lstlisting}

% {encoding}, [solution encoding]
In order to be of any use genomes must \Keyword\Target[\Term{encode}]%
[encoding]{solution encoding} the solutions they represent in a way the
applied genetic algorithm understands, making it possible to evaluate,
compare and evolve them.  \FIGure{ga} has already alluded to a potential
encoding, in which genomes are partitioned into several compartments
called \Term{attributes} or \Term{genes}.  An attribute has a name by
which it can be uniquely identified in a given genome.  Every attribute
encodes a different piece of information.  The type of this information
(the value of the attribute) is arbitrary (e.g. numbers, text, or any
complex data structure).  It is worth noting that many genetic algorithms
expect that the genomes in the population have the same set of attributes.

% {[GA hierarchy]}
The engine of \Menugene employs three kind of genomes: weekly menu plans,
daily menu plans and meals.  Each is bred in separate populations,
not allowing for mixins.  The relation between the different kinds of
genomes is \Keyword\Target[hierarchic]{GA hierarchy}, as depicted in
\Figure{mg} along with the corresponding attributes.  The point is that
\emph{the values of the attributes of all three levels of nutrition
are solutions themselves picked from an appropriate problem space}.
This is the distinctive feature, which we shall credit to \cite{TDK}
explicitly, making it possible to break down a complex problem into
several smaller ones with fewer dimensions (\Talk{divide \& conquer}).

%
% figs/mg.tex
%

\begin{NewFig}{mg}
\IncludeFig[0][0][trim=0 1mm 3mm 10mm]{mg}
\caption[Internal representation of menu hierarchies]%
	{Internal representation of menu hierarchies.  Ellipses stand
	for GA genomes, arrows indicate attribute-of relationships
	between.  Note that the arrows originating from the rectangles
	of populations are meant to signify that all genomes contained
	have the same set of attributes, only with different values.}
\end{NewFig}

% End of figs/mg.tex


% {genetic operators}
% [NOTE Math mode doesn't linebreak, which is kinda stupid I think.
The genetic algorithms of \Menugene have two \Keyword[means of evolving]%
{genetic operators} genomes.  One of them is the standard \Term{crossover},
which takes a pair of genomes (the parents) and results in a pair of new
ones (the children).  All attributes of the parents live on in the children,
but it is up to the crossover operator which child inherits a particular
attribute.  For example, crossing the genomes in \Figure{ga} may result in
two children having $\{$ attribute$_{\mathrm{N}1}$, attribute$_{12}$, \dots,
attribute$_{\mathrm{NM}}$ $\}$ and $\{$ attribute$_{11}$,
attribute$_{\mathrm{N}2}$, \dots, attribute$_{1\mathrm{M}}$ $\}$.

\Term{Mutation} is the other way (as far as \Menugene is concerned) to
renew a genome.  From the genetic algorithm's point of view by mutation
a single genome is replaced by another solution.  In \Menugene this is
implemented by asking some attributes of the genome under mutation to
acquire a new value.  How the attribute responds is a function of its
type, opening the scene for three alternatives.

% {[attribute types]}, [attr-pool]
\Tech{GA}-type \Keyword\Target[attributes]{attribute types} (L1 and
L2 only) run a genetic algorithm (which is private to the attribute),
and the new value of the attribute will be the best solution of this
GA can produce.  \Target[\Tech{POOL}]{attr-pool}-type attributes choose
their values from a preset static pool of genomes randomly.  In \Menugene
recipes are in a pool, because they do not change during generation.
(The system generates menus, not recipes.)  A \Tech{COMPOSIT} attribute
masters a set of subordinate attributes and asks each of them to renew
its value by whatever means it is capable of.  At the end \Tech{COMPOSIT}
master will choose the best solution presented by its subordinates.
This kind of relationship is useful when the problem space has sensible
(to humans) divisions, such as coldcut breakfast and brunch.

These mechanisms are stacked recursively (i.e. a \Tech{COMPOSIT} attribute
may have \Tech{COMPOSIT} subordinates).  \FIGure{relations} illustrates
all the meaningful combinations of relations between different attribute
types.  The pool of recipes are read from a \Link<database>, whose annotated
schemata is reproduced in the \Link[appendix]{database schema}.  The included
listing is intended as a starting point to grasp the conception of dietary
menus of \Menugene.

% {evaluation}
To determine which one is the best, solutions are \Keyword[\Term{evaluated}]%
{evaluation} and compared to each other.  Naturally, the evaluation
must reflect the \Link[consumer's requirements]{consumer objectives}:
\HisHer nutritional needs, dietary allowances, and taste.  The first
of these are taken care of by the first pass of the evaluation, the
\Term{fitness function}.  In \Menugene nutritional needs are expressed
by a \Term{constraint set}.  For a given nutrient a constraint specifies
its expected minimal and maximal amount in a solution, along with the
desired optimum.

% {[fitness calculation]}
A solution (weekly, daily menu or a meal) may have multiple constrained
nutrients, all of which the fitness function must \Keyword\Target[take
into account]{fitness calculation} (\Term{multi-objective optimization}).
First it \Link[decodes]{solution encoding} the solution to be evaluated,
and calculates the cumulative amount of the nutrients in question.
These figures are realized at the bottom of the \Link<GA hierarchy>
by genomes from a \Link[\Tech{POOL}]{attr-pool} representing recipes.
Each sum is used to sample a curve parametrized by the corresponding
constraint; an example is given in \Figure{fitness}.  The grand total
of these samples constitutes the \Term{fitness score} of the solution.

% {[rules]}
Per se the fitness score provides enough information for the genetic
algorithm to estimate the `goodness' of a solution.  Before that, however,
the remaining consumer objectives (dietary allowances and taste) are
accounted for.  Recall that these requirements state what dishes or what
combinations thereof the consumer \Either[should not][must not][would not]
eat.  \Menugene handles these within its \Keyword\Target<rules> subsystem.
A rule is essentially a pattern of solutions.  If the pattern matches
the contents of a solution (e.g. the items of a daily menu) its fitness
score is scaled with  factor associated with the rule.%
\footnote{We have intentionally omitted many details; more information
can be found in \File<src/MGLib/doc> of the source repository.}

It is possible to attach constraints and rules at any levels of the
\Link<GA hierarchy>.

% {running times}
% [TODO Tell them these numbers are sans DB load time?]
To speed up menu generation the engine employs some optimization
techniques, notably caching the sums in the \Link<fitness calculation>,
and the \Link<copy on write> manipulation of genomes.  \Keyword<Running
times> vary depending on the configuration of the genetic algorithms,
but it has proven capable of completing the task in five seconds for
daily menus, and in half a minute for weekly menus.
% The works >>>

%%%%%%%%%%%%%%%%%%%%%%%%%%%%%%%%%%%%%%%%%%%%%%%%%%%%%%%%%%%%%%%%%%%%%%
\NewChapter{Overview of the implementation}

\NewSection{Brief history} % <<<

\Topic{introduction}
Whenever someone meets a sizeable piece of software with the intention
of dwelling into its internals questions are naturally raised about
the events that has influenced its history in order to put its current
state into context, allowing for predictions.  The following account was
partly recovered from memory, and as such, at times it cannot be free from
personal tone.  Certainly, many episodes had to be omitted, save for the
most important ones.  Interested readers can learn more from the
\File<commitlogs>.

% {zeroth generation}
\Menugene started its existence as a scientific experiment of Bal�zs
Ga�l, an undergraduate student of information technology then, who
meant to explore innovative methods for computer aided production
of optimal dietary menus in 2002. All scientific work was done at
that time, including research of the literature, elaboration of
the multi-level organization of genetic algorithms, its application
to the problem space, and the design of the fitness function with
consultancy by dietary experts. The net result was a test program
written in Microsoft Visual C++ demonstrating the feasibility of the
idea and taking performance measurements. We call this software the
\Keyword<zeroth generation> \Menugene because later it became the
takeoff point of further development. The work was presented on the
2003 annual National Academic Student League (OTDK) \cite{TDK}.

% {[first generation]}
The zeroth generation \Menugene was incapable of anything beyond its
designated scope (namely performance testing) and was not adjustable for
other purposes. To remedy the situation, in January 2004 the author was
asked to join the development, and shortly after the \Keyword\Target%
<first generation> was born. The objective was to create \Link\Tech{GSLib},
a software library that would embody the optimization algorithm (above
all the handling of multi-level genetic algorithms) on the top of an
external generic GA library, \Prog{GALib}. We still have both of them,
although the function of \Link\Tech{GSLib} changed later; now that part
is called \Link\Tech{MGLib}.  The original intention was to make provision
for other problem spaces which could make use of MLGA optimization. Since
then this has become a minor point, and today the software concentrates
exclusively on dietary menu generation.

% {daemon}
One of the urgent tasks was to write a networked program we ended up
calling simply \Keyword\Link\Tech{daemon}, which would receive the
desired parameters of a menu over network, then as a server would
generate it using the new \Link\Tech{GSLib} library. Before that a
number of things awaited for being taken care of, therefore the first
period of the project was spent on design decisions and factorization:
class hierarchy, object relationships and recognized interfaces. Few
code fragments of the consolidated library from that time has survived
up to these days, but many of the concepts introduced then have remained.

% {infrastructural crisis}
By March 2004 \Link\Tech{daemon} had acquired some functionality: it
built the internal data structures essential for the menu generation
from the database, but \Link\Tech{GSLib} was not ready for that yet.
It would have been vital for the project that the library become
operational as soon as possible, but the early implementation had
\Keyword[begun to show its fragility]{infrastructural crisis}. It is
common knowledge that software components can usually be divided into
two broad categories: the ones (call them the \Term{engine}) that are
very specific to the task your program will eventually perform, and
those (the \Term{infrastructure}) whose reason to exist is to support
the engine, not knowing anything about the details of the problem.
Out of indolence the first generation \Link\Tech{GSLib} employed a
set of homegrown (and quite unsophisticated) classes for low-level
data structure handling, whose maintenance got burdensome due to
the increasing demands of the engine, the optimization algorithm.
The infrastructure either had to be got right or be superseded by
a third-party library.

% {[second generation]}, [Aapl], [TinyXml]
The quest for a definite solution took several weeks and an experimental
subproject (\Link\Tech{migraine}). Finally a new external, template-based
library, the \Target\Prog{Aapl} got merged into the source code repository,
and this marks the birth of the \Keyword\Target<second generation>.
With \Prog{Aapl} one could define and manage collections of arbitrary types
relatively easily, so our goal was achieved, it seemed. It felt so successful
that in a week another external library, \Target\Prog{TinyXml} landed in
the repository in order to take over XML output production (formerly it
had been done via unstructured \Code{printf()}s).

% {junkie}, {[Cordelia]}
These merges gave significant boost to development progress. Early in
June 2004 we could run the optimization algorithm of \Link\Tech{GSLib}
over random data via \Tech{randtree} (since then this program has
become part of the \Link\Tech{MGLib} test suite and it lives under
the name \File[mlga]{src/MGLib/test/mlga.cc}). Certainly we wanted to
generate dietary menus, and it was \Link\Tech{daemon} which connected
the algorithm with the nutrition database. While that program had a
developer-friendly command-line interface, the plan had always been
to have \Link\Tech{daemon} running silently in the background on a
server machine, and let a client application display the results in
a user-friendly manner. Thus \Keyword\Link\Tech{junkie} was born,
a web application, which communicated with \Link\Tech{daemon} over
network.  By August 2004 the system had become demonstratable, and
some months later it was coupled with \Keyword\Target\Prog{Cordelia},
the Department's lifestyle counselling system.

% {[avl]}, {MenuGeneAdmin}, {ontologies}
After these accomplishments development split into two branches.
One of them aimed at reengineering \Link\Tech{GSLib} to get rid of
\Link\Prog{Aapl}; it was necessary because few felt comfortable
with the compromises induced by the unnatural linkage of that library.
This effort (the \Keyword\Target\Tech{avl} branch), took a while to
mature, so in the meantime the other branch engaged in tasks that did
not depend directly on \Link\Tech{GSLib}. One of them was the creation
of \Keyword\Link\Tech{MenuGeneAdmin} for the dietary experts hired
for professional consultance, enabling them to remotely edit our
nutrition database. Another subproject that emerged at that time was
the exploitation of \Keyword<ontologies> to represent the knowledge
of a human dietary expert. Being done that would make it possible for
humans to configure the optimization algorithm using terms and concepts
more familiar to them than the \Link[solution-attribute model]%
{solution encoding} understood by the system.

% {[database migration]}
Lastly, it turned out to be unavoidable to \Keyword\Target[migrate]%
{database migration} and rework our nutrition database. Since day one
\Menugene had relied on the database of \Prog{Cordelia}, which was
sold and abandoned in the course of events. Its data had been derived
from sheets of a commercial vendor (\Prog{QuadroByte}) and it had been
managed by \Prog{Oracle} on an unattended box. The problems were that
doubts arose concerning the correctness of the information content,
and that nobody wanted to dig into \Prog{Oracle} and learn its advanced
capabilities. It was time to work out an appropriate database schema,
fill in with industry-quality content collected by the U.S.  Department of
Agriculture \cite{USDA}, and place it in the hands of \Prog{PostgreSQL},
which the team knew better. These subprojectes, ontologies and migration,
are still in flux, but definitely they are being worked on.

%[SKIP]
%
%% CORBA
%In late April 2006 \Link\Tech{daemon} (which was also renewed, hence the
%version bump) got checked in, and soon it featured a new interface for
%CORBA clients. \cite{Attee}
%
%% rules
%Having a solid foundation, it was timely to add a
%long-awaited feature, rule support (see p.~\pageref{rules}) in Sept 2006.  

% {[third generation]}
Meanwhile the \Link\Tech{avl} branch of the split was shaping, and evolved
into a complete, future-proof rewrite of the existing \Link\Tech{GSLib}.
The reason behind the main surgery is that one needs to break certain
psychological barriers to make large-scale incompatible changes. Once
the taboos are down you might be better off getting through all of
your long-standing intrusive ideas, because the next opportunity may
come too late. In case of \Menugene among others the object model, the
monitoring and error-reporting facilities, the XML writer underwent major
improvements, many test programs were added and lots of documentation was
written.  For a detailed list of accomplishments the reader is referred
to the \File{TODO}{src/GSLib/doc} file. In particular, it was then when
the engine was moved to its own library, \Link\Tech{MGLib}, leaving
all the infrastructural code in \Link\Tech{GSLib}. In mid-July 2006 the
branch was ready to take over the mainline, and the \Keyword\Target%
<third generation> came into existence, making it feasible to implement
some greatly anticipated features.

% {future}
\Keyword[At the time of writing]{future} (\today) most development power
is spent on the \Link<database migration>. The management is seeking for
commercial investors, and the eventual goal is to make enough profit to
get the project self-supporting. May we see whatever outcome, the author
is confident that future developers will have a solid foundation that
is worth building their innovations upon.

\begin{table}
\centering
\begin{tabular}{rrl}
Date	   & Subversion revision & Event				\\
\hline
2004-01-19 & r36  & \Link\Tech{GSLib} v1 (initial checkin)		\\ % *
2004-01-23 & r50  & large-scale OO reimplementation			\\
2004-02-23 & r99  & \Link\Tech{daemon} checkin				\\
2004-03-07 & r157 & \Link\Tech{daemon} takeover				\\
2004-03-11 & r173 & \Link[\Tech{daemon2}]{daemon} born			\\
2004-03-13 & r183 & \Link[\Tech{daemon2}]{daemon} builds menu trees	\\ % *
2004-03-19 & r199 & major rewrites (starting to take over)		\\
2004-05-16 & r237 & \Link\Tech{migraine}				\\ % *
2004-05-22 & r239 & \Link\Tech{GSLib} v2 (\Link\Prog{Aapl} checkin)	\\ % *
2004-05-28 & r289 & \Link\Prog{TinyXml} checkin				\\
2004-06-03 & r311 & \Tech{randtree} is functional			\\ % *
2004-06-06 & r314 & \Link\Tech{mdb}					\\
2004-07-28 & r456 & \Link[\Tech{daemon2}]{daemon} network mode		\\
2004-08-06 & r462 & \Link[\Tech{junkie2}]{junkie} born			\\ % *
2005-04-05 & r511 & \Link\Prog{Cordelia} integration			\\ % *
2006-04-07 & avl  & \Link[\Tech{daemon3}]{daemon}, \Link\Tech{ttc}	\\
2006-04-24 & avl  & \Link[\Tech{daemon3}]{daemon}: provision for CORBA	\\
2006-07-13 & r625 & \Link\Tech{GSLib} v3 transition complete		\\ % *
2006-09-06 & r662 & \Link<rules> integrated				\\
2006-10-04 & r669 & \Link\Tech{rulez} born				\\
\end{tabular}
\caption{Milestones of project history}
\end{table}

\begin{table}
\centering
\def\ifusedtreshold{999}
\begin{tabularx}{\linewidth}{r|X}
VASS�NYI Istv�n		& PostgreSQL, general advisor,
			  academic contact			\\
GA�L Bal�zs		& original idea, project executive,
			  business contact, chief evangelist	\\
V�GS� Bal�zs		& \Link\Tech{MenuGeneAdmin}		\\
MAYOR Zsolt		& initial \Link\Tech{daemon} code	\\
CZIG�NY Attila		& CORBA interface for \Link\Tech{daemon}\\
SZENTE Zsuzsanna	& ontologies				\\
HERCZINGER Viktor	& PostgreSQL, system administration	\\
the author		& pretty much all and everything else	\\
\end{tabularx}
\caption{Past and present project participants}
\end{table}
% Brief history >>>

\NewSection{Qualitative goals} % <<<

% {extreme programming}
% [SKIP] "the only truly important product of the system development process
% is code"
% [SKIP] "There is no Big Design Up Front."
\Topic{introduction}
\Menugene was not developed with any particular methodologies in mind
because none of the team members had been educated in these matters
of software engineering.  Nevertheless we cannot say the process was
driven by individuals' inconsistent actions, either.  In retrospect we
can identify a set of values that we took quite seriously, cutting out
any others from the subconscious.  This is similar to \Keyword\Target%
<extreme programming> (XP, see \cite{XP}), an agile software development
methodology which is expressly suited for research projects having
unstable requirements.  It turns out many of the properties described
below play an important role in extreme programming as well.

\NewwSection{Maintainability} % <<<

We postulate \emph{no part of a system lasts forever}: one day or another
it will be necessary to improve, fix, or investigate any given code fragment,
no matter how still it used to be.  To save the codebase from becoming
an obstacle in its own development we strongly discouraged writing code
on the assumption that once it is finished, it can be forgotten.

% {reasonable factorization}
In practice, one way to ensure maintainability was paying special attention
to \Keyword<reasonable factorization>.  Whenever a new function was added,
in the reviewing cycle we routinely examined: \Talk{Is it the appropriate
place for that function?  Wouldn't the implementation be simpler elsewhere?
Or maybe is it worth splitting it up for better understanding?}  If so, we
tried hard to reach a reassuring conclusion.

% {elegant interfaces}
Related to code organization is the question of interfaces.  As we all
know, an entry point lies between the caller and the callee, their
interaction depending on the encoding and decoding of arguments and
return values, and the \emph{mental} overhead of these processes can be
significant (then painful).  We realized this, and as a countermeasure
we always sought to come up with \Keyword<elegant interfaces>.  Sometimes
we found compromises unavoidable, because what callers appreciate is not
necessarily comfortable for the callee.  Other times we had to resort to
code refactorization to be able to move, split, merge, generalize, or
specialize interfaces, all because it appeared mental overhead would be
less the other way.  A set of well-pronounced guidelines on this topic
can be found in \cite{dietlibc} by the author of ``glibc on diet''.

Gradual refactorization (both code- and interface-wise) is a key element
of extreme programming.  While it might be regarded as trivial day-to-day
practice, we decided this task deserved more responsibility.  If for no
better reason, because it can break existing code considerably (which we
did \Link[several times]{brief history}), but we have learnt being courageous
(one of the principal XP-values) pays off.

% {consistent coding style}
% [SKIP] common language -- common ownership
XP also advocates good communication, which we (unknowingly) sustained
by growing accustomed to a \Keyword<consistent coding style>.  \emph{%
Developers express and interpret (communicate) program behavior by writing
and reading code}, and they code the same thing very differently, in the
terms of their personal vocabulary.  We thought, at the cost of some soft
restrictions we could spare a lot of mental overhead, because reviewers
would not be forced to change their mindset when working on components
written by someone else.  Our coding style covered code formatting, markups,
naming conventions, and the usage of language features.  Once it had been
documented, but later we saw it was better to let people accommodate,
rather than treating them as children who must obey their father's rules.

% {[proper documentation]}
% [SKIP] improve language skills
Finally, our professional experience had taught us not to neglect
\Keyword\Target<proper documentation>.  Following our \Link[postulate of
maintainability]{maintainability}, during the course of development we
settled down on the principle \emph{nothing in the codebase should be
regarded as private asset} which need not be documented because it is
shielded by publicly accessible layers.  We attempted to make careful
note of every detail which might not be obvious by just looking at the
source, including expected entry conditions, side effects, relations with
other functions (contributing to the access protocol of the component
they are part of), and everything known to require further attention
(TODO).  Of course, there was the danger of watering down the source
by excessive commentary%
%%%%%%%%%
\footnote{Not a big threat as for today's standards.}.
%%%%%%%%%
For us another utility of documentation was to gain feedback: \Talk{Does the
behavior I have coded really make sense?} It occurred to us an idea is less
likely to contain flaws if it can be phrased convincingly both in a spoken
and in a programming language.
% Maintainability >>>

\NewwSection{Correctness} % <<<

% {automated testing}
% [TODO] \cite{some software engineering book (SSLD?)}
Besides the qualitative feedback picked up by writing documentation it was
apparent we needed some quantitative measurements as well.  Books on modern
software engineering call for \Keyword<automated testing>, an idea we
adopted wholeheartedly for its promise of thoroughness and of saving human
resources.

% {torture testing}
One can automate his tests by several means.  \Keyword[Knuth]{torture
testing}, for example, to verify the robustness of his \Prog{TeX}
compiler crafted a special input file along with the expected
output \emph{manually} in the hope that the presence of bugs in the
implementation would be visible either by a distraction in the output
or by an unwelcome crash: \cite{Knuth}

\begin{quote}
The idea is to construct a test file that is about as different from a
typical user application as could be imagined. Instead of testing things
that people normally want to do, the file tests complicated things that
people would never dare to think of, and it embeds these complexities in
still more arcane constructions.
\end{quote}

Knuth went as far as demanding no program shall claim itself a \TeX\
implementation until it passes this test.  He concluded: ``This method
of debugging, combined with the methodology of structured programming
and informal proofs (otherwise known as careful desk checking), leads
to greater reliability of production software than any other method
I know.'' While we support this view, for \Menugene we needed a test
method that relies less on the elaboration of test vectors.

% {unit testing}
An appealing alternative was \Keyword<unit testing>, so prevalent among
Java developers that every book on Java programming, it seems, feels
obliged to mention \Tech{JUnit} at least%
%%%%%%%%%
\footnote{Giving the impression that Java does not know about any
other testing methods---a statement we cannot accept faithfully.}
%%%%%%%%%
(see \cite{Java} for example).  Its application is quite diverse, but we
can make a couple of observations.  First, unit testing (as a methodology)
does not let testing become an afterthought, done (or forgotten altogether)
after all user requirements have been met, because test cases and production
code must be developed in parallel%
%%%%%%%%%
\footnote{Even more, in a test-driven development pattern the test case
for a unit is created \emph{prior to} writing production code.}
%%%%%%%%%
Consequently, in the codebase there cannot be
untested units.  Exactly what constitutes a unit is largely unspecified
\cite{IEEE:unit_testing}, but the assumption of a one-to-one mapping between
units and classes appears to be customary.

% {integration testing}
Most certainly we wanted testing to be an integral part of our software
development lifecycle. On the other hand the tendency of unit testing
\Talk{if each unit feels good, the whole program feels good} raised
concerns about \Keyword[inter-unit relationships]{integration testing}.
By writing endless test cases one risks missing the forest for the
trees, because operations of one object may \emph{inadvertently} affect
the state of other objects. Prime examples are global variables, shared
resources in general and concurrency.

% {[stress testing]}
% [TODO] Reference literature for fuzzing?
In the end what we have come up with may not be very different from some
interpretation of unit testing. Our test suites consist of standalone
programs, which are not built around any standard framework (although we
managed to factor out some common code).  Since our \Link<build system>
understands their invocation it is a matter of keystrokes to launch a test
(or an entire test suite), and see if there is any regression. A test
program tries to stress the implementation by use case basis, sometimes
covering a subset of a class, other times more than one class. The test
subject is \Keyword\Target[put under stress]{stress testing} by feeding
it with \emph{bulk}, \emph{random}%
%%%%%%%%%
\footnote{Actually, pseudo-random.  The seeds are always saved for
the sake of reproducibility.}
%%%%%%%%%
input for a \emph{long} time. The input is generated on the fly
in such a way that frequently values are chosen near the limits
of the input domain of the variables. IEEE refers to this method
as \Term{boundary-value analysis} \cite[chap.~5]{SWEBOK}. For the
input is ensured to be acceptable (though meaningless) the subject
must not crash (see \Section{robustness}).

Our line of thinking was the longer the subject is exposed the more likely
we will hit a bug---if there is any.  The advantage of the randomization
is that we need not design test vectors to the minute details; it is enough
to set the main directions, and leave the rest to chance.  This technique
is called \Term{fuzzing} in security engineering, and its usage is naturally
a compromise, but a good one at that in our assessment.

% {[co-testing]}
Stress testing (combined with measures to ensure robustness) performed
well at demonstrating the behavior of the test subject is not incorrect,
but its capability of gathering positive evidence is limited, because
often the correctness of the results can only be checked indirectly.
For this reason we advanced the method a little, and termed the result of
our efforts \Keyword\Target<co-testing> for it is based on the simultaneous
execution of two processes in a client-server manner, the client being
the test program, and the server being the test subject. During testing,
the client sends requests to the server, but \emph{the operation is
carried out by both parties}. At regular intervals the subject is asked
to return detailed information about its state. As this information is
internally tracked by the test program too all the time, they can be
compared to see if the subject has been malfunctioning. IEEE mentions
this method under the name \Term{back-to-back testing}.

% {qbot} TODO needs \Tech on \marginpar
It is important to emphasize the independence of the implementation
of the test program. In our case not even the programming language is
shared, yielding additional benefit of having insight into the problem
the subject tries to solve from a different point of view . In an
\Keyword[other project]{qbot} we experimented more with co-testing, and
finally established a framework for trying the robustness and conformance
of untrusted programs. \cite{qbot}
% Correctness >>>

\NewwSection{Robustness} % <<<

% {defensive programming}
One of our priorities was to write programs that \Keyword[behave well]%
{defensive programming} even under unexpected circumstances: if the input
is intended (documented) to be acceptable (from the perspective of the
callee) then, be it however extreme, it must be processed correctly;
otherwise the error must be signalled to the caller very clearly, As we
have seen in \Section{correctness}, much has been invested in the testing
suite to probe this property of the implementation. Eric S. Raymond (ESR)
characterizes robustness as ``the child of transparency and simplicity''
\cite{ESR}. We have already touched upon these dimensions of software
quality in \Section{maintainability}; now we will review the measures
we have taken to prevent bogus input \emph{silently} producing bogus
output\footnote{This is not to be confused with GIGO (garbage in, garbage
out), which is said when nonsensical (but acceptable) input about is
fed to a well-meaning, correct program.}.

% {appropriate data types}
ESR says ``One very important tactic for being robust under odd inputs
is to avoid having special cases in your code.'' \cite[\Ibid]{ESR}
We translated his advice into practice by making a good deal of
special cases impossible.  More closely, whenever feasible we declared
variables with \Keyword[constrained types]{appropriate data types}
(like \Code{unsigned} and \Code{enum}%
%%%%%%%%%
\footnote{%
The GNU C Compiler also has an \Code{__attribute__((not_null))}
extension, meaning a pointer declared such cannot be \NULL.  It is yet
to be investigated, though propagation rules may wreck havoc on its
usefulness.  PostgreSQL 8.x has a more flexible mechanism to attach
\emph{custom} constraints to data types via \Code{CREATE DOMAIN}.
This feature fit very well in our database schemata.}).
%%%%%%%%%
Not only this gave a chance to the compiler to do static type
analysis and catch misuse, but the point is to \emph{confine the domain
of the variables to the range that actually makes sense in that context}.
A frequent use case is storing size or quantity, which obviously cannot
be negative.  Having not restricted their domain, they must be checked
before usage---a special case we sought to fend off by ESR's words.

% {[consistency checkpoints]}
Choosing the appropriate data type is a good starting point, but we
knew it cannot work miracles, for example when the constraint is too
complex to be formulated within the facilities of the type system of
the target language%
%%%%%%%%%
\footnote{%
This is the case with Perl, which does not even distinguish between
numbers and strings. In exchange it offers excellent validation
capabilities.},
%%%%%%%%%
or when the constraint is not known until run-time. As another layer of
defense we strewed the codebase with internal \Keyword\Target<consistency
checkpoints> everywhere we could think of. Similar in spirit to database
constraints, these checkpoints try to spot inconsistencies in the state
of the program as early as possible, and once one is found, they bring
on outright termination%
%%%%%%%%%
\footnote{%
The Linux kernel developers take the same approach, except that they
seldom panic the kernel in consideration of fault tolerance. Instead
they let the error signal propagate back to user space.}.
%%%%%%%%%
For implementation we used both \Code{assert(3)}-style short statements
and separate sanity checking routines. After a while we came to using
these safeguards as complementary formal documentation, and begun adding
them purely for that purpose, even if they were known to be redundant.
The price to be paid was the CPU-time spent on consistency checking,
causing an order of magnitude slowdown. To cancel the effect, our
\Link<build system> can be \Link[configured]{CONFIG_DEBUG} to skip all
the checkpoints, or just the most expensive ones of them.

% {input validation}
Robustness is a requirement of secure software. While \Menugene is not
specifically security-sensitive (save for \Link\Tech{junkie}, the web
interface, which \emph{is}), buying their agenda \cite{SecProg} is
always a good idea, because one day any piece of code can end up in
a context where security is a chief concern. One lesson people have
(hopefully) learnt from web application development is the necessity
of (program) \Keyword<input validation>. \cite{OWASP} exemplifies
a number scenarios when invalid input leads to incorrect behavior in
subtle ways. It is important to realize that it does not make difference
whether the input is invalid on (malicious) intention or just by mistake%
---the end result is the same unless this vector is taken care of properly.

% {error checking}
The textbook rule of thumb says you must also account for errors reported
by subroutines (e.g. checking the return value).  While it may sound too
pedantic to bring up such a triviality in this document, visible evidence
suggests developers still perceive \Keyword<error checking> as a bothersome
and unfruitful exercise%
%%%%%%%%%
\footnote{%
Exceptions may help to hide this problem, but they will not take you much
farther; see the next paragraph.}.
%%%%%%%%%
It is not that we were after illegible code for the sake of purity,
but in our experience trading reliability for convenience is not a
profitable deal in the long run.

% {graceful failure}
We have already alluded to various error handling strategies we employed
in \Menugene: for example, halt on internal consistency error (bug) or
on critical resource outage (i.e. insufficient stack space, or object
allocation failure).  The repertoire is broad; however, the point we
made was to \Keyword[fail gracefully]{graceful failure}: do not crash
the program, and try to clean up before exit.  The latter clause may
seem unrelated to the topic of this section, but an adhering component
will appear to be more robust by not leaving mess behind even if it finds
itself in trouble.  Looks human psychology pervades software engineering.
% Robustness >>>

\NewwSection{Serviceability} % <<<

\emph{Designing for failure may be the key to success} is the title of an
interview \cite{Lindsay} conducted with Bruce Lindsay, IBM DB2 architect
and RDBMS guru.  Sections \Section*{correctness} and \Section*{robustness}
showed how \Menugene is prepared to meet bugs and other failures.
This section describes the built-in facilities that help developers and
operators track down the source of failures.

% {[information gathering]}
In such a situation it is probably information what the maintenance
team appreciate the most: information about what the system had
been doing before the error occurred, details about the context
etc.  Run-time \Keyword\Target<information gathering> is facilitated
by the \Link\Tech{gsmon} subsystem, which bears some resemblance
with the \Tech{syslogd} daemon.  Scattered in the codebase, just like
\Link<consistency checkpoints> there are calls to \Link\Tech{gsmon}
that inform about the progress of execution, current state, warn
before entering and after leaving critical sections, or about anything
the developer thought \emph{might} be worth making record of some day.
The logged information is filtered by the subsystem, and only the relevant
pieces are presented.  The source of the information (which component,
function emitted it) is preserved as well as the call stack at the time
the message was logged (with a little aid).

% {error stack}
Components often fail due to the failure of another component. This case
the error propagates back in the dynamic scope of the program.  The error
could be logged by every component involved in the chain of events.  While
\Link\Tech{gsmon} is perfectly suitable for this usage, we chose to add an
\Keyword<error stack> manager, \Link\Tech{gserr} to assist handling the
chain of errors%
%%%%%%%%%
\footnote{In Java, we talk about exception chaining.}.
%%%%%%%%%
In a typical use case when an error is encountered it is added to the
queue, and the control is returned to the caller (indicating the failure).
Depending on the error handling strategy of the caller, it may give up
too, retry the operation, or just acknowledge the error and proceed.
Not all of these cases needs the error to be reported.  \Link\Tech{gserr}
can clear the queue either reporting or just forgetting its contents,
depending on the judgment of the caller.  This way the full chain of
events can be retrieved if necessary, but they will not flood the logs
if they are harmless.

% {service interfaces}
Both logging and the error queue are active in normal operation mode.
\Menugene also has \Keyword<service interfaces>, which come into play
in actual debug sessions. Some of the interfaces function as hooks.
These are placeholder NOP instructions which you can set a breakpoint
on, allowing, for example, to trap exceptions.  (This is not possible
otherwise in C++.)  Other service interfaces exist to instruct various
components of the system to produce extra debugging information (like 
statistics, for profiling).
% Serviceability >>>

\NewwSection{Care} % <<<

In a way care can be regarded as a meta-value: having high
standards of the values discussed so far infers care about
the quality of the software overall. Extreme programming has
added respect as the last of its own values (the original
proposition did not have it).  XP practitioners respect their
work by pursuing high quality and looking for the best design
for the solution at hand.

% prototyping
Unfortunately quality does not come for free. Often we found ourselves
spending days or even weeks on interface design, refactorization, and
\Keyword<prototyping>, coding several possible variations to choose the
best one.  Done with that in some cases we managed to integrate prototypes
into our \Link[co-testing framework]{co-testing}, saving materia that
otherwise would have been consigned to oblivion.
% Care >>>
% Qualitative goals >>>

\NewSection{Development environment} % <<<

\NewwSection{Platform} % <<<

% {Linux}
Before the author joined \Menugene had been developed and run on
Microsoft Windows. Shortly after the team shifted to \Keyword<Linux>,
and has remained with it ever since. The reason behind the move was
simply that we had more experience with Linux, and we wanted to support
that platform anyway. Currently the system neither compiles nor runs
under Windows, but porting should be fairly trivial, because care
was taken not to overuse platform-specific features. Beside the two
mainstream desktop operating systems, no other has ever been considered.
% Platform >>>

\NewwSection{Programming languages} % <<<

The bulk of \Menugene is written in C++, mixed with some embedded SQL,
surrounded by Perl and PHP programs, and glued by shell scripts.

% {C++}
The choice of \Keyword<C++> was predetermined by the external low-level
genetic algorithm library, \Tech{GALib}, which was also written in C++,
and replacing that library was out of question%
%%%%%%%%%
\footnote{%
The project management pled to its alleged widespread scientific usage.}.
%%%%%%%%%
Since the engine of \Menugene and \Tech{GALib} need to cooperate
closely, choosing another language would have introduced unacceptable
risks of instability. C++ is a usual compromise between execution speed,
development speed, and portability. For \Menugene the first one is the
deciding factor, for genetic algorithms are truly CPU-intensive.

% {[migraine]}
As noted above, C++ is usually a compromise. Many feel dissatisfied
with its misfeatures and unpredictable behavior (see \cite{Critique}
for a well-founded case against C++). \cite{LiteFAQ,GotW} list
\emph{hundreds} of obscure situations when the language acts far off
the expectations---not one of them we had the pleasure to deal with.
In an attempt to back out of the jumble of C++ we created a short-lived
branch, \Keyword\Target[\File{migraine}{src}]{migraine}, to probe into
Objective~C. The subproject was canned soon, but the gathered experience
bore heavy influence on the memory management subsystem of the second
generation \Link\Tech{GSLib}.

% {Perl}
The need for a higher level language was stressing. As an alleviating
resolution we incorporated \Keyword<Perl> to write auxiliary and helper
applications, such as GUI visualizers, like \Link\Tech{rulez}. Perl is
very good at handling data structures, functional programming and
object oriented programming%
%%%%%%%%%
\footnote{Despite the near-complete lack of special syntax for OOP.}.
%%%%%%%%%
Its characteristics makes it ideal not only for prototyping, and the
author expects to see an increasing portion of \Menugene being written
in Perl, if the development of the system is to be continued once.
% Programming languages >>>

\NewwSection{Compilers} % <<<

% {gcc}
Contrary to popular misconceptions the C++ compiler of the GNU Compiler
Collection (\Prog{gcc}%
%%%%%%%%%
\footnote{%
By the acronym we will refer both to the GNU C and GNU C++ compilers.
See \cite[p.~3]{gccman} for other meanings of the acronym.})
%%%%%%%%%
is not the only one available on Linux. A serious alternative is the
Intel C/C++ compiler (\Prog{icc}), which is said to generate faster,
superior code. However, we decided to stick with \Keyword\Prog{gcc},
which we understood very well, and reserved \Prog{icc} as an option
for later evaluation. \Prog{gcc} has a wide variety of vendor-specific
extensions which we sought to exploit aggressively (grep the source for
``gcc-specific'' to see the whereabouts). As neither compiler-neutrality
nor purity was an objective explicit dependency upon these extensions
only made life easier for us.  Arguably, it is a bad idea to get into a
vendor lockin deliberately, but in this special case there is a way out:
the \Prog{icc} is rumored to be compatible with \Prog{gcc} to such an
extent that even the Linux kernel has no problem with it. \cite{ICC}
% Compilers >>>

\NewwSection{Build system} % <<<

By \Term{build system} we refer to the framework that utilizes your
compiler toolchain to automatically compile source files and produce
an executables. (See \cite[p.~114]{SWEBOK} for a better, more abstract
definition.) Self-contained IDE:s (like Microsoft Visual Studio and
Eclipse) come with their own build systems, storing related settings in
\Term{project files}.  Unlike with the \Prog{gcc}, we definitely did
not want to depend on a particular IDE to work on the system---people
are different, after all, and an editor that fits one's needs may be
a pain to use for others.

% {make}
Instead the build is controlled by GNU \Keyword\Prog{make}%
%%%%%%%%%
\footnote{%
It must be noted that there exists many other build systems, such as
\Prog{cons} and the infamous \Prog{jam}.  It is only that the extended
capabilities of GNU \Prog{make} made it a satisfactory solution for us.}.
%%%%%%%%%
Making use of the macroing capabilities of the GNU version (\Code{\$(call
)}), all the building logic is sourced out to one specific location
(the \File{BUILD}{src} directory).  Developers can usually get on by
copying an existing \Path{Makefile} and adjusting a few variables as
documented in \File{Makefile.vars}{src/BUILD}.
% Build system >>>

\NewwSection{Software configuration managers} % <<<

% [TODO] Mention darcs?
A software configuration manager (SCM) is a program that, put very simply,
tracks the history of the codebase and allows team members to share their
improvements with each other.  When a change is done the developer must
notify (\Term{commit}) the SCM.  \cite[chap.~7]{SWEBOK} has a comprehensive
review of all related concepts.

% {CVS}
IEEE attributes critical role to SCM in the software maintenance process.
\cite{IEEE:maintainability} It is no coincidence that the codebase of
\Menugene has been under version control since the very beginning.
Initially \Keyword\Prog{CVS}, the de facto tool of not-for-profit
development efforts was responsible for SCM. It was neither particularly
featureful nor reliable; especially its inability to follow file and
directory renames was disappointing.

% {Subversion}
When it become apparent that \Prog{CVS} loses commits we decided it
was time for departure. A potential successor was \Prog{Monotone}, a
distributed SCM supporting disconnected (offline) operation. This mode
was quite welcome by the author, who did not have Internet connectivity
at home at that time. We considered \Prog{Monotone} briefly, but the
team rejected it on the grounds of being too different from what we were
used to. Eventually \Keyword\Prog{Subversion} took over \Prog{CVS}
at the same time when the \Link<third generation> of \Link\Tech{GSLib}
was merged. The transition period lasted no longer than a week, and
at the end, much to our reassurance we managed to preserve the history
of the codebase.
% Software configuration managers >>>

\NewwSection{Developer tools} % <<<

Out of respect to developers' freedom we did not discipline what CASE
tools are to be used for everyday duties, except when agreement was
absolutely necessary (e.g. the choice of SCM).  As a matter of long-%
established preference the author keeps on using \Prog{Vim} for editing,
\Prog{gdb} for debugging and \Prog{ctags} for generating source code
cross-references.  Other team members settled on \Prog{Eclipse} and are
content with it.

% {[homegrown utilities]}
An often neglected segment of software construction is the creation of
\Keyword\Target<homegrown utilities> that automates repeated tasks (for
example, disassembly of a dump file) or makes the developer's life easier
otherwise.  We did not make trouble of it and wrote new tools whenever
necessity came in the same vein as any other code.  The \Link[next section]%
{BITS} provides a detailed listing and description of these tools.
% Developer tools >>>

\NewwSection{Database managers} % <<<

% {Oracle}
Part of the \Menugene system is an \Keyword\Prog{Oracle} database
manager, which provides efficient access to nutrition data for other
system components. (See \Figure[sysarch]{executive summary}.) The
database schema and contents dates back to days when the author was not
affiliated with the project yet, when \Menugene was planned to be a
subsystem of \Prog{Cordelia}, the Department's lifestyle counselling
system. (A weak linkage has been \Link[fixed]{Cordelia}, but only for
demonstration purposes.)

\Topic{criticism}
It is obvious now, \Prog{Oracle} is an \Keyword[overkill]{criticism}.
Its capabilities are greatly under-utilized, and the database schema
has stagnated beyond all bearings. \Menugene has proven not to be a
data-intensive application anyway, as its data supply scarcely exceeds
a few megabytes.

% {PostgreSQL}
In response to the first and second lines of criticism, the project is
drifting away from the world's leading vendor of information management
software towards \Keyword\Prog{PostgreSQL}. One of its key advantages
that led to the decision is its remarkably comprehensive and readable
documentation%
%%%%%%%%%
\footnote{%
Though compared to the API:s and manuals of \Prog{Oracle} anything may
seem `excellent'.}.
%%%%%%%%%
The transition is also a good opportunity to rethink and (at last)
document the database schema---a long-due debt to repay.
% Database managers >>>

\NewwSection{Documentation} % <<<

% {source code commentary}
\Menugene does not have any \emph{formal} documentation, largely
because there was no up front design in the first place; this document
is written in the hope to \Link[fill the gap]{docaim}. This is not to
say there is not an explanatory word in the codebase, that would be
against our \Link[qualitative goals]{proper documentation}. Rather,
most of the information is in the \Keyword<source code commentary>.
There was an early attempt to make it compatible with \Prog{doxygen},
which would then produce nice HTML sheets just like \Prog{Javadoc},
but the effort has lost interest. In the \Path{doc} subdirectories
there are some complementary files for the record of notes not fitting
elsewhere. The draft \Prog{PostgreSQL} database schema is kept in
\File<src/daemon/doc/postgre.sql>. The description of various database
objects (\Code{TABLE}s, \Code{DOMAIN}s, \Code{VIEW}s etc.) are also
available on-site in \Code{COMMENT} blocks, so SQL development environments
can access them.

% {TODO files}
Bug tracking (\Term{defect tracking}) is also handled informally.
Once we had a \Prog{Bugzilla} to put bugs on public exposure, but
it turned out not to help productivity.  What we needed instead
was a registry of missing and broken features ordered by relative
importance to know what areas call for attention.  This list is
maintained in the \Keyword[\File{TODO}{src/GSLib/doc}]{TODO files} file.
By recalling past version of this file via the SCM, one can estimate
the accomplishments and limitations of the system at any given time.
% Documentation >>>

\NewwSection {Quality assurance tools} % <<<

% {memory-related bugs}
Several tools have been put in service to assure the \Link<correctness>
and \Link<robustness> of \Menugene. While they differ in specific uses,
all of them are employed to unveil \Keyword<memory-related bugs>. A common
detrimental side-effect of these tools is the significant slowdown they
cause, making their application difficult during \Link<stress testing>.

% [efence], [ccmalloc]
\begin{itemize}
\item	\Target\Prog{efence} is a library that transparently replaces
	all routines of the \Code{malloc(3)} family to catch \emph{%
	buffer overruns} as soon as they happen.  The technique is solid,
	but wasteful: even a single byte of allocated memory will occupy
	kilobytes.  (The formula is not linear.)  Worse, on Linux the size
	of user address space will drop down to a fraction of the normal
	limits.  This means the program will not be able to claim more
	than half gigabytes of memory, no matter how much is available.

	Nevertheless, the help of \Prog{efence} has been invaluable.
	Years ago we forked a version under the name \Tech{ElectricTense}
	for private enhancements.  Since then it has grown completely
	independent, and today it is part of \Cite\Prog{qbot}.
\item	\Target\Prog{ccmalloc} is another transparent library, which
	we primarily used to detect \emph{memory leaks}.  The library
	can tell where the unreleased piece of memory was allocated
	in the source, hinting at the faulty data path.
	% [SKIP] OCI leaks like hell
\item	\Prog{valgrind} is a virtual machine which observes program
	behavior on the machine instruction level, and draws inference
	from that.  We used it to spot \emph{references to uninitialized
	variables}.
	% [SKIP] doesn't work under PaX
\item	\Link\Tech{mdb}:
	One of our homegrown utilities to \emph{profile memory consumption}.
\end{itemize}

% [SKIP] These are C-specific (except valgrind).
Java developers will notice their choice of language offers these services
out of the box. While there is no getting around it, it must also be noted
the tools listed above are not the king achievements of the 21st century
realm of C.  For one, the copyright of \Prog{efence} dates back as far as
1987---eight years before the burst of Java.
% Quality assurance tools >>>
% Development environment >>>

\NewSection{The codebase} % <<<

\NewwSection{A walk through the repository} % <<<
In this section we will review the contents of the \Menugene repository.
We will concentrate on role and functionality, allowing for few technical
details.  These can be followed up at the referred pages, or in the source
code itself if the component is not discussed further in this document.

\begin{itemize}
\item	\Repo{BUILD}{src}\par
	This directory contains the common files needed to
	\Link[build]{build system} the system.

	\BumpBMLevel
	\def\cwd{src/BUILD}
	\begin{itemize}
	\item	\Repo{eclipsize}\cwd\par
		This script is invoked by \Link\Code{make eclipse}
		to get the source browsing capabilities of the
		\Prog{Eclipse} IDE fully functional.  It does
		so by linking header files to another place
		in the file system.  \Link\Code{make xclean} undoes
		the effect, making the repository pristine again.
	\item	\Repo{Makefile.vars}\cwd\par
		This file is sourced by every \Path{Makefile}s.
		It describes the build environment, such as the
		desired compiler to use.  These parameters are
		intended to be customizable to fit local needs.
	\item	\Repo{Makefile.macros}\cwd\par
		Sourced internally by \File{Makefile.vars}{src/BUILD},
		and describes the actual operating system commands to
		execute in order to build a part of the system.
		It should be general enough to be applicable
		unmodified to any Unix-like systems.
	\item	\Repo{Makefile.rules}\cwd\par
		This file is also sourced by every \Path{Makefile}.
		It consolidates all the information from the
		\Path{Makefile}s described above, and eventually
		tells \Prog{make} when, how and what to build.
	\end{itemize}
	\KnockBMLevel

\item	\Repo{GALib}{src}\par
	Complete source code of \Prog{GALib}, the external low-level
	genetic algorithm library \Menugene relies on.  It is included
	in the repository to be at hand if the system happens to need
	a feature not provided by the upstream version.  It has been
	brought up-to-date with newer compiler standards, and the source
	layout has been rearranged to be more consistent with the rest
	of the repository.

\item	\Repo{GSLib}{src}\par
	This directory contains the source code and test suite of
	\Link\Tech{GSLib}, the infrastructural library of \Menugene.
	In more precise terms, in this library reside the code of
	the \Link<object model>,
	the generic complex \Link[data structures]{collections},
	the \Link[XML output driver]{XML driver},
	the \Link<logging>, and
	the \Link[error reporting]{error handling} subsystems.

\item	\Repo{MGLib}{src}\par
	In this directory are the source code, documentation and the
	test suite of the \Link\Tech{MGLib} library, the engine of
	\Menugene, responsible for the generation of optimal menu plans;
	this is where the optimization algorithm lives.  This library
	is not an abstraction layer above \Link\Tech{GSLib}, because
	client applications need to interfere with both of them.

\item	\Repo{migraine}{src}\par
	An early attempt to \Link[reimplement]{migraine} the optimization
	method in Objective~C.  By now it has lost all of its functions,
	but we believe the idea is worth keeping in mind.

\item	\Repo{perl}{src}\par
	This is the location of all sort of Perl modules \Code{use}d
	by Perl programs all over the repository.

	% [GSDB]
	\BumpBMLevel
	\def\cwd{src/perl}
	\begin{itemize}
	\item	\Repo{Adios}\cwd\par
		\Tech{Adios} is the author's bundle of custom Perl modules
		for random utility purposes.  It is not maintained as part
		of \Menugene; what is in the repository is merely a copy
		of the (unreleased) upstream.
	\item	\Repo{CoTest.pm}\cwd\par
		Provides common functions for \Link<co-testing> parties.
		It helps in establishing a communication channel between
		the tester and the subject, and in comparing the results
		of the parallel execution.
	\item	\Target\Repo{GSDB.pm}[GSDB]\cwd\par
		This is an abstraction layer which makes accessing
		the \Menugene database conceptually more natural
		for the programmer.
	\item	\Repo{Rules.pm}\cwd\par
		Prototype implementation of the \Link[rule matching]{rules}
		function of \Link\Tech{MGLib}.  In part, it is used for
		testing.
	\item	\Repo{MyPrima.pm}\cwd\par
		This is a collection of widgets and subsystems that
		extends the functionality of the \Prog{Prima} GUI
		toolkit, which is required by some of the \Menugene
		programs.  Among the additions are a drag-and-drop
		manager, an undo manager, a flexible scrollbar manager,
		and a new layout manager.
	\end{itemize}
	\KnockBMLevel

\item	\Repo{daemon}{src}\par
	There is the source code of \Link\Tech{daemon}, a client program
	that takes user criteria, loads some data from the database and
	generates appropriate menu plans with the help of \Link\Tech{MGLib}.
	It has a network interface, making it suitable to be used as a
	mid-level tier behind a user-friendly presentation layer.

	\BumpBMLevel
	\def\cwd{src/daemon}
	\begin{itemize}
	\item	\Repo{postgre.sql}{\cwd/doc}\par
		This is the draft of the new \Prog{PostgreSQL} database
		schemata of \Menugene.  The database that the system has
		today unfortunately lacks documentation.  Until the
		replacing database manager takes place this draft can be
		consulted to get an impression about the conceptual design
		of the current schema.
	\end{itemize}

	The following files appear in many other parts of the codebase,
	so it is worth casting a few words about them.

	\begin{itemize}
	\item	\Repo{.depend}\cwd\par
		This file is generated by \Link\Code{make dep} and
		used by the \Link<build system> of \Menugene to
		track dependencies between source and object files.
	\item	\Repo{.ccmalloc}\cwd\par
		This is the configuration file of \Prog{ccmalloc},
		one of our \Link[quality assurance]{ccmalloc} tools.
	\item	\Repo{.suppress}\cwd\par
		Configuration file of \Prog{valgrind}.
		Both \Path{.suppress} and \Path{.ccmalloc} are used
		to instruct the associated tools to try to keep out
		false positives from their failure report as much
		as possible.
	\item	\Repo{.gdbinit}\cwd\par
		Initialization file of \Prog{gdb}, the author's choice
		of debugger.  The commands in \Path{.gdbinit} make
		the debug session more convenient and more efficient.
	\end{itemize}
	\KnockBMLevel

\item	\Repo{junkie}{src}\par
	Source code of \Link\Tech{junkie}, the chief end-user
	web interface of \Menugene.  It relays user input to
	\Link\Tech{daemon} through network and presents the
	results in a neat form.

	% [rulez]
\item	\Target\Repo{rulez}{src}\par
	\Tech{rulez} is a GUI program for assembling \Link<rules> for
	\Link\Tech{MGLib}, guiding it in the selection of menu plans
	based on user taste and other preferences.  Originally, it was
	intended to be just developer tool to create test vectors for
	the associated functionality, but we believe its user interface
	has been polished up to the level of meeting not-so-high end user
	expectations, like drag'n'drop and unlimited depth undo.
	\FIGure{rulez} demonstrates the program in operation.

% [BITS], [asciitree], [coxi], [baggy], [subborn], [mdb], [ttc]
\item	\Target\Repo{BITS}{src}\par
	This is the developer's private quarters.  The code in this
	directory -- our \Link[homegrown tools]{homegrown utilities} and
	other asset -- is required neither for building nor running the
	system, but can be of value when developing it.  Demonstration
	screenshots can be seen at the \Link[appendix]{screenshots}.

	\BumpBMLevel
	\def\cwd{src/BITS}
	\begin{itemize}
	\item	\Target\Repo{asciitree.pl}[asciitree]\cwd\par
		Text-mode XML visualizer.  It reads XML files and
		draws flat ASCII diagrams showing the hierarchy of
		XML elements.  This tool is used to inspect the XML
		output of \Link\Tech{daemon}.
	\item	\Target\Repo{coxi.pl}[coxi]\cwd\par
		This is a GUI XML browser, useful at exploring larger
		XML structures than \Tech{asciitree} is suited to.
		We had evaluated a number of mature XML browsers before
		writing \Tech{coxi}, but all of them felt so heavyweight
		that we decided they were not worth the disk space for
		our modest needs.
		% [SKIP]
		% \FIGure{coxi} shows a screenshot of our program.
	\item	\Target\Repo{baggy.pl}[baggy]\cwd\par
		Along the lines of \Tech{coxi}, this is the browser of the
		\Menugene database.  It displays the \Link[relationships]%
		{attribute types} between solutions, attributes and solution
		sets.  \Tech{baggy} was our first GUI tool, and in part it
		served as a pilot project probing into the GUI potentials
		of Perl---with salient success.
	\item	\Repo{dolly.pl}\cwd\par
		Downloads the complete \Menugene database and saves
		it in a compact binary file which \Tech{baggy} and
		\Link\Tech{rulez} can read, so they can operate
		disconnected from the database manager.
	\item	\Target\Repo{subborn.pl}[subborn]\cwd\par
		\Tech{subborn} is a \Prog{Subversion} repository browser,
		with special focus on convenient history traversal.
		In a mouse click it lists what the currently selected
		directory contained at the previous revision, and with the
		help of the \Prog{Vim} editor it can depict the differences
		between two revision of the same file in unmatched quality.
		To our astonishment, we found no standalone applications
		among the mainstream \Prog{Subversion} GUI clients that
		had similar features.
		% [SKIP]
		% \FIGure{subborn} shows the program in action.

	\item	\Repo{PROTO}\cwd\par
		\BumpBMLevel
		Prototypes for various parts of the system used to live
		here.  As time passed and they had fulfilled their purpose,
		most of them have been removed from the repository.  The
		only one remaining is \Repo{cow.pl}{\cwd/PROTO},
		modeling the implementation of the optimization algorithm
		of \Link\Tech{MGLib}.
	\item	\Repo{demos}{\cwd/PROTO}\par
		The diagrams in this directory have been generated by
		\File{cow.pl}{\cwd/PROTO}, and have been used to examine
		and demonstrate workings of the optimization algorithm
		as implemented in the prototype.  It has been planned
		to add a real-time monitor to \Link\Tech{daemon} with
		similar diagrams on display.  This work is yet to be done.
		\KnockBMLevel
	\item	\Repo{TOYS}\cwd\par
		Developer playground.  The code fragments here are not
		meant to do anything useful, they are just samples of
		how to accomplish a particular task.

	\item	\Target\Repo{mdb}\cwd\par
		\BumpBMLevel
		\Tech{mdb} is a small library for memory profiling.
		In run time it logs all memory operations (\Code{malloc()},
		\Code{realloc()} and \Code{free()}) to a binary file which
		can be analysed with \Repo{mdbstat.pl}{\cwd/mdb} later.
		In this data one may discover patterns in the memory
		usage of her program, and associate them with operations
		being executed at that time, helping code optimization.
		\KnockBMLevel
	\item	\Target\Repo{ttc}\cwd\par
		\BumpBMLevel
		Total tracer is a C++ add-on tracing function and method
		calls during a program's lifetime.  Similar to \Tech{mdb}
		these events are recorded in a binary log file, which is
		decoded by \Repo{ttc.pl}{\cwd/ttc}.  The output is listing
		of what functions in what order from where were called during
		the tracing.  Naturally, this list can grow quite long,
		but the decoder takes great pains making it as compact
		and readable as possible.
		\KnockBMLevel
	\end{itemize}
	\KnockBMLevel
\end{itemize}
% A walk through the repository >>>

\NewwSection{Dependencies} % <<<
\Menugene makes extensive use of external tools and libraries.
The following packages are sufficient to build and run all of
the components of the system.

% [myOCI]
\begin{adjustwidth}{-1cm}{-1cm}
\def\ifusedtreshold{999}
\bigskip\noindent
\begin{tabularx}{\linewidth}{>{\raggedright\arraybackslash\hsize=.3\linewidth}X|>{\hsize=.7\linewidth}X}
Package					& Remarks
\\\hline
GNU \Prog{make}				&
	any recent version will do, but it must be the GNU variant
\\
a minimal POSIXy environment		&
	required by the \Link<build system>
\\
the \Prog[GNU C++ Compiler]{gcc}		&
	v3.4 is what we are using; v3.0 will not do, v3.3 might do,
	v4.0 is known to generate incorrect code
\\
\Prog{Perl}				&
	either v5.6 or v5.8 will do but the former lacks a few modules,
	which might need to be installed separately then
\\% [myOCI]
the \Prog[Oracle Client Interface]{OCI}	&
	either v9.2 (9{\itshape i} series) or v10.2 (10{\itshape g} series)
	will do; for the least trouble we recommend \Target[our]{myOCI}
	\File[stripped OCI distribution]{ARCHIVE/Oracle}
\\
the \Prog{Prima} Perl GUI toolkit	&
	required by \Link\Tech{rulez} and \Link\Tech{subborn}; since Linux
	distributors seem not to offer binary packages, we have rolled our
	\File[own]{ARCHIVE/Prima}, which is as easy to install as unpacking
	a \Tech{tar} archive
\end{tabularx}
\end{adjustwidth}

\bigskip
Also, one may want to obtain the packages below in case she
intends to carry on the development of \Menugene and put our
\Link[in-house tools]{BITS} in working order.

\begin{adjustwidth}{-1cm}{-1cm}
\def\ifusedtreshold{999}
\bigskip\noindent
\begin{tabularx}{\linewidth}{l|>{\hsize=.7\hsize}X|>{\raggedright\arraybackslash\hsize=.3\hsize}X}
Package & Description & Needed by
\\\hline
  \Prog{gobjc}
& the GNU Objective~C Compiler
& \Link\Tech{migraine}
\\\Prog{zsh}
& the Z shell (Unix command line interpreter)
& \Link\Tech{mdb}
\\\Prog{DBD::Oracle}
& \Prog{Oracle} backend for DBI, the Perl Database Interface
& \Link\Tech{GSDB}
\\\Prog{Gtk}
& Perl bindings to the \Prog{gtk} GUI toolkit library
& \Link\Tech{baggy}
\\\Prog{Math::Random}
& Perl module generating random deviates;
  used to generate test vectors
& co-testers, \File{cow.pl}{src/BITS/PROTO},
  \File{cross.pl}{src/BITS/PROTO}
\\\Prog{Math::Combinatorics}
& Perl module enumerating combinations and permutations;
  used to generate test vectors
& \File{cross.pl}{src/BITS/PROTO}
\\\Prog{HTML::Parser}
& One-pass HTML processor module for Perl;
  used to read XML files
& \Link\Tech{asciitree}
\\\Prog{HTML::Tree}
& HTML parser module for Perl;
  used to read XML files
& \Link\Tech{coxi}
\\\Prog{Date::Parse}
& Perl module interpreting textual representations of dates
& \Link\Tech{subborn}
\\\Prog{Graphviz}
& Graph rendering software package
& \File{relations.pl}{src/MGLib/doc},
  \File{cow.pl}{src/BITS/PROTO},
  \File{cross.pl}{src/BITS/PROTO}
\end{tabularx}
\end{adjustwidth}
% Dependencies >>>

\NewwSection{How to build it} % <<<
% {[daemon]}
Building the executables of the system is relatively straightforward.
The first step is to adjust the settings in \File<src/daemon/daemon.pc>
under the heading ``Ora connectinfo''.  Based on these settings knows
\Keyword\Link\Tech{daemon} where to find and how to log in the database
manager.  We have not made these values run-time configurable because
our server has never changed.  Nevertheless it would be a reasonable
improvement.

% {OCI}, <$ORACLE_HOME>
It is wise to verify the \Var{$ORACLE_HOME} environment variable
is set correctly.  This variable belongs to the \Keyword\Prog{OCI}
libraries, and is supposed to point to a directory where C header
files and shared libraries can be found underneath.  If you have
our \Link[OCI distribution]{myOCI} installed, \Code{source}ing
\Path{/usr/local/bin/oracle_env.sh} will set the variable correctly.

% {configuration}, [config.h], <[CONFIG_DEBUG]>
There are some other \Keyword<configuration> options and tunable
parameters in \Target[\File<src/GSLib/include/gslib/config.h>]%
{config.h}, but we figure that their present values are just right.
It might be desirable to change \Target\Var{CONFIG_DEBUG} of
\File<src/BUILD/Makefile.vars>, which controls whether to build
debugable or production object code.  See the files themselves
for the complete list of available settings.

% {make}
Now \Keyword\Prog{make} is ready to build the executable parts of \Menugene.
Invoked from the root of the repository (\Path{src}) it will take care of
everything automatically. Components can also be built individually by
issuing the command from their directory. The code should compile without
warnings when \Code{CONFIG_DEBUG=full}. Below that debug level (\Code{half}
or \Code{none}) the ever-cautious \Prog{gcc} will take notices about possible
instances of uninitialized variables; these warnings have all been verified
to be false positives and can safely be ignored.  The following operations
(\Term{targets}) are available:

% <make all>,
% <[make dep]>, <make .depend>
% <make <source..o>, <make <program>>
% <make tags>, <[make eclipse]>
% <make clean>, <[make xclean]>
\begin{itemize}
\item	\Code{make} or \Make{all}:\par
	Just build everything in the current directory.
\item	\Target\Make{dep} or \Make{.depend}:\par
	Recalculate object file dependencies.  \Prog{make} will into
	all source files (\Term{sources}) to see what other files
	(\Term{includes}) do they refer to.  This is necessary for the
	build system to determine whether an object file (\Term{object})
	is out of date and needs recompilation.  You should \Code{make
	dep} whenever a new include is added to any of the dependencies.
\item	\Make{<source>.o}:\par
	Build the named object if it does not exist or any of its
	dependencies has modified since the last build.
\item	\Make{<program>}:\par
	Likewise.
\item	\Make{tags}:\par
	Index all source files with \Prog{ctags} for quick crossreference.
\item	\Target\Make{eclipse}:\par
	Clone header files where \Prog{Eclipse} finds them.
	This is necessary because of a \Prog{gcc} extension
	(\Code{-remap}) that \Prog{Eclipse} does not understand.
\item	\Make{clean}:\par
	Delete all compiled files from the current directory
	and downwards.
\item	\Target\Make{xclean}:\par
	Delete all reproducible files.  Implies \Code{make clean}.
\end{itemize}

The test suite of \Link[\Tech{GSLib}]{gslib:test} and \Link\Tech{MGLib}
are not administered from the top-level directory; see their sections
to learn how to run them.

% {deployment}
The eventual target of \Prog{make} is the executable of \Link\Tech{daemon}.
Its \Keyword<deployment> is very simple: copying that single file to
any box where \Prog{OCI} has been installed will suffice.  Deploying
the other components of the \Menugene can be just as easy as well, but
so much depends on the actual operating environment that the matter is
best left to experienced system administrators.
% How to build it >>>
% The codebase >>>

%%%%%%%%%%%%%%%%%%%%%%%%%%%%%%%%%%%%%%%%%%%%%%%%%%%%%%%%%%%%%%%%%%%%%%
\NewChapter[Implementation details]{GSLib}

% Intro <<<
\Topic{introduction}
\Tech{GSLib} is the infrastructural library that both \Link\Tech{MGLib}
and \Link\Tech{daemon} relies on.  It is a pool of functionalities not
directly related to the problem \Menugene tries to solve---optimal dietary
menu generation.  Its name (``Genetic Solver Library'') originates from
the \Link[time]{first generation} when we thought bundling everything into
a single library would be maintainable.  Since then the infrastructural
needs have grown, and the engine has been granted its own library,
\Link\Tech{MGLib}.

\Warning
% {template instantiation}, {header organization}
\Tech{GSLib} is written in C++, and (though not for the faint of heart)
makes heavy use of templates. Generic programming (templates) is a
complex addition to modern C++, with startling syntax and peculiar
semantics. One of its toughest aspect is the point of \Keyword<template
instantiation>. Basically, the question is where in the source
code instantiate templates in order to reduce compilation time and
object code size. \Prog{gcc} offers several options \cite[Template
Instantiation]{gccman}. Since neither of the factors mentioned was
our concern, we decided to follow the Borland model and ``Do nothing
[extraordinary]. Pretend g++ does implement automatic instantiation
management.'' In a nutshell, in the Borland model template definitions
must be visible to all sources that instantiate them, and the compiler
emits object code for every instance for each file being compiled. In
our case the definitions are in the files of \File<src/GSLib/inline>,
which are \Code{#include}d whenever necessary. The drawback of
this approach is that header files become interdependant upon each
other---a phenomenon well-known to C programmers. To cut the long
story short, we have come up with an \Keyword[organisation]{header
organisation} in which header files are split into multiple files
-- declarations, definitions, macros, prototypes -- and every one
of them only \Code{#include}s those parts that are really needed,
ordered such that the dependency cycles break open.  This mechanism
is transparent to \Path{.cc} source files.  The commentary in
\File<src/GSLib/include/gslib.h> presents an illustration of the
problem and the solution.

% {macros}
Originally, templates were introduced into C++ to save developers from
the `evil' C preprocessor, and increase the type safety of the language.
It is ironic that the C++ standard committee seems to have fallen between
two stools in their effort. It is our persuasion that the preprocessor
has a place in the developer's toolbox, and we did not refrain from using
it, despite its deteriorated reputation. As a result, \Tech{GSLib} has
dozens of \Keyword<macros>, which are essential for the workings of the
library.  A quick survey of \File{gslib/macros.h}{src/GSLib/include}
will reveal that very few of them could possibly be replaced by templates.
In fact, many of the macros are empolyed to circumvent the all too
disciplining nature of C++.  Definitely, this is a case when good
intentions fire back on consumers.

% {macros.h}
\Keyword[\File{macros.h}{src/GSLib/include/gslib}]{macros.h} is our
store of general macros not belonging to any specific subsystem of
the library.  They define shorthands for common expressions, help
in defining classes, simplify casting etc. We cannot review here
all the 69 of them, but the documentation in the file should give
adequate coverage.
% >>>

\NewSection{Object model} % <<<

% {[GSObject]}
\Keyword\Target\Tech{GSObject} is the core object model of \Tech{GSLib}.
The model defines a set of conventions that every \Tech{GSObject} must
comform to, and another set of methods by which any comforming object
can be manipulated.  The semantics and the implementation may be familiar
from other frameworks; in particular, many ideas were borrowed from Java
and Objective~C.

\NewwSection{Basic facilities} % <<<

% {CObject}
\Keyword\Code{CObject}%
%%%%%%%%%
\footnote{All identifiers defined in the library are in the \Code{gslib}
namespace.  For the sake of readability we omit the prefix in the text.}
%%%%%%%%%
is the ancestor of all \Tech{GSObject}s.  The common operations are
propagated via inheritance (whenever possible), and the conventions
are enforced via abstract methods (whenever possible, see \Link[usage]%
{gsobj:usage}).  

% {reference counting}
% [SKIP] The very same technique is used in the Linux kernel.
First and foremost what a \Tech{GSObject} `knows' is the semi-automatic
management of its own lifecycle.  It is well-established that C++ does
not reclaim dynamic memory automatically.  The problem is that it is
left upon the shoulders of developers to find out when it is safe to
\Code{delete} an object.  This is not trivial, because many other objects
may hold reference to the named one.  \Tech{GSObject} helps to alleviate
the problem by providing little-overhead means of explicit \Keyword<%
reference counting>.  The principle is simple: a counter in every
\Tech{GSObject} maintains the number of referrers to that object.
Referrers express their (loss of) interest in a \Tech{GSObject} by
manipulating that counter.

% <retain()>, <[release()]>, <is_protected>
\begin{itemize}
\item	\Fun{retain}{}:\par
	Tells the receiver \Tech{GSObject} that the caller got hold of
	a reference to it.  We speak about \emph{joint ownership} if an
	object has more than one referrers.
\item	\Target\Fun{release}{}:\par
	Informs the receiver that one of its referrers has ceased caring
	about it.  When its last referrer has gone, the object is destroyed.
\item	\Var{is_protected}:\par
	Setting this field \Code{true} indicates that the object has a
	permanent referrer, which will stay there no matter how many times
	\Code{release()} is called.  Consequently, the reference counter
	will not be allowed to reache zero, and the protected object will
	not be destroyed then.
\end{itemize}

\Topic{release() policy}
A newly created \Tech{GSObject} is assumed to be owned by its creator.
When an object is destroyed, it must explicitly \Code{release()} every
dynamically allocated \Tech{GSObject}s that it has reference to. Inline
(non-dynamically allocated) \Tech{GSObject}s must not be \Code{release()}d
by their container object, and at the time of destruction they must not
be referred by other objects.

% {ownership transfer}
During its lifetime ownership of a \Tech{GSObject} \Keyword[ownership
transfer]{may change}, for example when it is passed to a setter method.
Technically, exactly nothing happens then, but semantically it implies
that one reference of the caller to the object will belong to the
called method from now on, as if the callee \Code{retain()}ed and the
caller \Code{release()}d the passed object. It is not specified in
what circumstances the ownership of a passed object is transferred, it
depends entirely on the called method. For setter methods the usual
convention is to take ownership, while getters keep their ownership of
the returned object.

% {protection}
Sometimes the forceful takeover may be \Keyword[unacceptable]%
{protection}.  For example, if the caller would continue to use
the passed object, but somehow it ends up destroyed by the callee.
This is when \Code{is_protected} can help.

% {duplication}
Besides reference counting \Tech{GSObject} has other things to offer with
respect to object \Keyword<duplication>.  An object can be in \Term{has}
relation with other objects, which in turn may \Term{have} other objects
etc.  When making copy of an object it is desirable to specify how deeply
the the hierarchy should be duplicated, and from which level should it be
a clone of the original.  The distinction between clones and duplicates
is clear: if \Code{A} is a clone of \Code{B} then changes to \Code{A}
are visible in \Code{B}, while if \Code{A} is a duplicate then \Code{B}
is unaffected (though initially they have the same contents).
\FIGure{dup} will clarify the meaning of \Term{depth}.

%
% figs/dup.tex
%

\begin{NewFig}{dup}
\IncludeFig{dup}
\caption[The effect of \Code{dup()}]%
	{The effect of \Code{dup()} with depths 0 and 1 on a three-level
	object hierarchy.}
\end{NewFig}

% End of figs/dup.tex


% <ctor()>, <[dup()]>, <[copy()]>, <is_shared()>, <unshare()>
\begin{itemize}
\item	\Fun{ctor}{mate, depth}:\par
	Every comforming \Tech{GSObject} is required to have a copy
	constructor with a prototype like this.  The \Code{depth}
	parameter specifies how many levels of the contents of
	\Code{mate} should be duplicated; above that level contents
	are cloned.
\item	\Target\Fun{dup}{depth}:\par
	Returns with an object of the same type as the receiver.
	\cite{LiteFAQ} calls it a \Term{virtual constructor}.
	The new object is initialized with the receiver.
	\Code{depth} has the same meaning as at the copy constructor.
\item	\Target\Fun{copy}{depth}:\par
	The same as \Code{dup(depth $-$ 1)} for positive values of
	\Code{depth}, otherwise the receiver is returned.
\item	\Fun{is_shared}{}:\par
	Returns whether the receiver has multiple referrers.
\item	\Fun{unshare}{depth}:\par
	\Code{dup(depth)} the receiver if it \Code{is_shared()}.
\end{itemize}

% <GSOBJ_{COPY,SET,ZERO,CLEAR}()>
\Topic{more operators}
\Code{dup()}, \Code{copy()}, and \Code{unshare()} return pointers
of type \Code{CObject *}.  All three of them have a capitalized pair
(\Code{Dup()}, \Code{Copy()}, \Code{Unshare()}), which behave exactly
like the lowercase versions, except that they return upcasted pointers.
In addition to these operators \Tech{GSObject}s can be moved around
with the macros \Fun{GSOBJ_{COPY,SET,ZERO,CLEAR}}{}.  They are
described in \File{GSObject/macros.h}{src/GSLib/include}.

% {interfaces}, [gsobj:interfaces]
Finally, a comforming \Tech{GSObject} \emph{may} choose to respond to a
series of other \Keyword\Target[requests][interfaces]{gsobj:interfaces}.
Implementing them is not compulsory---\Tech{CObject} only defines them
to make it syntactically valid to invoke the methods in question of
\emph{any} \Tech{GSObject}, regardless whether it actually implements
them.  This can be considered a workaround to a fundamental flaw in the
`object orientation' of C++.

% <[compare()]>,   <[IComparable]>
% <[serialize()]>, <[ISerialisable]>
% <[print()]>,     <IPrintable>
\begin{itemize}
\item	\Target\Fun{compare}{mate, flags}:
	(interface \Target\Class{IComparable})\par
	Asks the receiver to compare itself to another object
	from the same class, honoring the semantics laid out in the
	\File[\Code{compare} namespace]{src/GSLib/include/GSObject/%
	definitions.h}.
\item	\Target\Fun{serialize}{options, base}:
	(interface \Target\Class{ISerialisable})\par
	Asks the receiver to give a \Link\Tech{GSXml} representation
	of itself.  \Code{options} is a collection of arbitrary options
	whose meaning is up to the receiver.  \Code{base}, if defined,
	shall be the parent of the XML node of the XML representation
	of the receiver class.
\item	\Target\Fun{print}{stream}:
	(interface \Class{IPrintable})\par
	Asks the receiver to give a byte representation of itself, and
	print it to \Code{stream}.  Details of this representation are
	unspecified.  At the moment \Tech{GSLib} uses XML exclusively,
	but you may choose any encoding you wish.
\end{itemize}

\Topic{other interfaces}
There are other interfaces that a \Tech{GSObject} may implement,
which are not part of the core \Code{CObject} class.  See \File%
{GSObject/definitions.h}{src/GSLib/include} for further possibilities.

% {usage}, [gsobj:usage]
At the end let us examine how to \Keyword\Target[define][usage]{gsobj:usage}
a class comforming to the \Tech{GSObject} model.  As mentioned earlier
in the section many constraints cannot be expressed with the inheritance
mechanism of C++. For this reason we need to use macros \emph{within}
the class definition.  Unfortunately, this practice tends to confuse
the source indexing feature of advanced IDE:s.  The following listing
illustrates the procedure.  For background, please see the description
of the macros involved.

% TODO \textbf keywords
\begin{lstlisting}[style=C++]
class SomeClass:
	/* GSObjects must be a descendant of CObject. */
	public CObject
{
public:
	/* Required by GSOBJ_DEFINE_METHODS(). */
	GSLIB_DEFINE_TYPE_THIS(SomeClass);

public:
	/* Declares the copy constructor as required by the
	 * GSObject object model. */
	GSLIB_DECLARE_CTOR_COPY(SomeClass)

public:
	/* Defines common GSObject methods which cannot
	 * be propagated via inheritance. */
	GSOBJ_DEFINE_METHODS();

public:
	/* The following items are optional. */
	/* Declares compare(), if you wish to provide
	 * a real implementation. */
	GSOBJ_DECLARE_COMPARE();

	/* Declares serialize(). */
	GSOBJ_DECLARE_SERIALIZE();

	/* Declares print(). */
	GSOBJ_DECLARE_PRINT();
};
\end{lstlisting}
% Basic facilities >>>

\NewwSection{Copy on write} % <<<

\Topic{what is CoW}
There are situations when having multiple \emph{private} copies of the
same object is desired.  This could be achieved easily by creating and
initializing the object, and \Link\Code{dup()}ing it as many times as
required.  Then changes to one duplicate of the object would not affect
the others, but the cost in terms of system memory may be intolerable.
We faced this very problem when implementing our \Link[genetic model]%
{solution encoding} in \Link\Tech{MGLib}, and our solution is based on
the obsevation that often the requirement of having a private copy can
be downgraded to ``give me a shared reference to the object which stays
as it is, and give some means by which I can make it entirely my own as
necessary comes''.  In less informal terms this technique is called
\Term{copy on write}.

% {protocol}, [cow:protocol]
The \Keyword\Target[protocol][protocol]{cow:protocol} is as follows: every
interested party gets the same reference to the object (let us call it
a \Term{cow}).  The parties are registered as \Term{watchers}.  They may
query the object as long as it remains unchanged.  When a watcher needs to
change the object the other watchers are notified beforehand, so they can
\Link\Code{dup()} it.  At the end there will be two copies of the object:
the original one in the private posession of the watcher which wanted to
change it, and a duplicate, being shared among the remaining watchers.
Of course the duplicate will have one fewer watchers than the original
object had.  \FIGure{cow} depicts the course of events.

%
% figs/cow.tex
%

\begin{NewFig}[1.0cm]{cow}
\FigPair{cow}%
[1]	{Initial state. \Tech{cow} is shared among four watchers.
	Solid lines represent \Term{has} relationships.}
[2]	{The fourth watcher indicates its desire to have a private copy
	of \Tech{cow}.  \Tech{cow} begins to notify the watchers about it.
	The first one \Tech{dup()}s \Tech{cow}, and \Code{release()}s the
	original object.  Dashed lines represent message routes.}
\FigPair{cow}%
[3]	{The notification of \Tech{cow} reaches all watchers.}
[4]	{Final state.  \Tech{cow} is owned by the fourth watcher
	exclusively; the other watchers share the duplicate of
	\Tech{cow}.}
\caption{The process of copying on write}
\end{NewFig}

% End of figs/cow.tex


% <[notify()]>, <GSOBJ_DECLARE_NOTIFY()>
\Topic{notify()}
For the \Target[notification mechanism]{notify()} to work watchers
must implement the \Tech{INotifiable} interface (declared with
\Fun{GSOBJ_DECLARE_NOTIFY}{}).  This interface is supposed to
\Link\Code{dup()} the \Tech{cow} if necessary, and replace the
watcher's reference to the original object by the new one.

% {cows}
A \Keyword[cow]{cows} object must be a descendant of the class \Class{CCow},
which contains the registry of watchers and the code for the propagation
of change notifications.

% <add_watcher()>, <del_watcher()>, <notify_watchers()>, <cow_depth>
\begin{itemize}
\item	\Fun{add_watcher}{referrer}:\par
	Adds \Code{referrer} to the list of watchers, which will be
	informed when the object is about to change.
\item	\Fun{del_watcher}{referrer}:\par
	Removes \Code{referrer} from the watcher list of the cow.
	\Code{referrer} remains a referrer---no \Link\Code{release()}
	is implied.
\item	\Fun{notify_watchers}{originator}:\par
	Issued by a watcher before it alters the state of the cow.
	By the time the method returns, the cow is guaranteed to have
	become private to the caller.
\item	\Var{cow_depth}:\par
	Specifies how deep the duplication of the cow should be on write.
	The default is to make the deepest copy possible.
\end{itemize}
% Copy on write >>>

\NewwSection{Containers} % <<<

\Topic{why containers}
C++ is not a fully object-oriented language.  Its primitive data types
(such as \Code{int}) cannot be handled as objects, therefore they cannot
be incorporated into the object model without some extra undertaking.
The ultimate goal is to be able to handle these primitives as if they
were true \Tech{GSObject}s.  In \Tech{GSLib} the issue can be handled by
wrapping the primitives in \Term{container} templates as \Term{payload}.
Some programming language call this technique \Term{boxing}.  The
containers are valid \Tech{CObject}s: they are refernce counted, and can
be passed to methods that only accept \Tech{CObject}s.  In addition, they
are making provision for the voluntary \Link[interfaces]{gsobj:interfaces}
of \Tech{GSObject}.  Containers are crucial for the implementation
\Link[collections]{GSAvl} in \Tech{GSLib}.

\Warning
% {container templates}
Data types are divided into four \Keyword[categories]{container templates}:
\Tech{Atom}, \Tech{Struct}, \Tech{CObj}, and \Tech{CCow} (not to be
confused with the \Code{CCow} class).  All of them has a corresponding
container template which can be used for wrapping.  Categorizing data
types was necessary because the containers must know at least that
much about the data types they are handling, and C++ has no syntax for
selecting between code fragments based on template arguments (much like
\Code{#if arg ==  #elif }).  Container classes are unrelated in the
object-oriented sense, because the signatures of their methods vary with
the data type the container wraps.

% <[GSOBJ_CONTAINER()]>
% <TContainerAtom>, <TContainerStruct>, <TContainerCObj>, <TContainerCCow>
% <[CContainerVaradic]>
\begin{itemize}
\item	\Target\Fun{GSOBJ_CONTAINER}{category, data_type}%
	%%%%%%%%%
	\footnote{%
	Actually, this macro and the \Code{TContainer{Atom,CObj,CCow}}
	classes have another argument, which specifies the type of the
	constant pointer if \Code{data_type} is a pointer.}:\par
	%%%%%%%%%
	Expands to a sequence instantiating the appropriate container
	template embedding \Code{data_type}.  This is preferred over
	spelling out the instantiation because it hides the line noise.
\item	\Tpl{TContainerAtom}{data_type}:\par
	For types you wish to pass by value when \Either[setting][retrieving]
	the payload of the container.  Typically, \Code{data_type} will be
	\Code{char}, \Code{unsigned}, or some pointer.
\item	\Tpl{TContainerStruct}{data_type}:\par
	The payload is communicated by address (nevertheless it is stored
	in the container in its entirety).

	\Warning
	Containers wrapping \Code{Atom}ic and \Code{Struct}-like types
	can \Link\Code{compare()} and \Link\Code{print()} themselves
	(that is, their payload), but this is subject to the availability
	of properly-typed \Code{compare_it()} and \Code{print_it()}
	functions, which carry out the actual work.  Many functions
	are supplied for the most primitive types (\Code{unsigned},
	\Code{double}, \dots), but we could not prepare for uncountably
	many user-defined types.  Thus, for example if you wish to use
	the \Link\Code{IComparable} interface of a \Code{Struct} container,
	it will be necessary to extend the \Code{compare_it()} set of
	functions with a new overload which knows how to deal with the
	\Code{Struct}.
\item	\Tpl{TContainerCObj}{data_type}:\Warning\par
	Wraps a \Tech{CObject}-derivate in a container.  Doing so makes
	sense when the container \emph{type} is passed as an argument
	to a template which asks for a container specifically.	In cases
	like this the concrete container type has to be passed, since the
	container templates have no common abstract container ancestor.
	\Link\Code{compare()} etc. are relayed to the payload directly,
	so the container will be comparable \emph{iff} the payload is.
\item	\Tpl{TContainerCCow}{data_type}:\par
	A descandent of \Code{TContainerCObj}, so everything written
	about it apply to this container as well.  In addition, the
	container becomes a watcher of the payload cow, therefore
	it responds to \Link\Code{notify()} events.  Upon its receipt
	the payload is \Link\Code{dup()}ed \AndOr replaced according
	to the \Link[protocol]{cow:protocol}.
\item	\Target\Code{CContainerVaradic}:\Warning\par
	A container whose type of payload can be changed in run-time.
	This class is not a template, and is typically used to define
	container-like fields of non-template classes.  Its greatest
	shortcoming is that it cannot be extended to handle arbitrary
	payload types.
\end{itemize}

% {accessing the payload}
What is left from containers is the question of \Keyword[setting and
retrieving the payload]{accessing the payload}.  It is unavoidable that
containers of different kinds behave differently, but much trouble was
taken to get them `do the right thing' consistently.

% [null value], <null_v()>
% <assign()>, <retrieve()>, <replace()>, <payload>
\begin{itemize}
\item	\Fun{ctor}{payload}:\par
	The payload can be set when the container is being constructed.
	If the container is constructed via the default constructor,
	its payload will be initialized to some (category-dependant) null
	value.
\item 	\Fun{null_v}{}:\par
	Returns the \Target\Term{null value} for the type of the payload.
	Null values are intended to represent undefined payloads.
	Unfortunately it is only reliable for \Tech{CObj} and \Tech{CCow}
	containers (for which the null value is simply \NULL).
\item	\Fun{assign}{new_payload}:\par
	Replaces the current payload of the container with
	\Code{new_payload}.  For \Tech{Atom}ic and \Tech{Struct}
	containers this method boils down to a C assignment.  \Tech{CObj}
	containers pay attention to \Link\Code{release()} the old payload.
	Ownership of \Code{new_payload} is taken.  \Tech{CCow} containers
	also start watching it.
\item	\Fun{retrieve}{}:\par
	Returns the current payload.  \Tech{CObj} containers retain
	the ownership of the payload.
\item	\Fun{replace}{old_payload *, new_payload}:\par
	Like \Code{assign()}, but the previous payload is returned,
	whose ownership is given away.
\item	\Var{payload}:\par
	Alternatively the payload can be accessed directly.
	Sometimes this is less cryptic, but it may hurt the
	interchangeability of containers.
\end{itemize}
% Containers >>>
% Object model >>>

\NewSection{Logging} % <<<

% {[gsmon]}
\Topic{purpose}
\Tech{GSLib} and the upper layers use logging to inform the user about the
current activites of the system: what it is doing, why is it doing that,
how it has interpreted the input, etc.  This information is a significant
\Link[development aid]{information gathering}.  The goal of the
\Keyword\Target\Tech{gsmon} subsystem is to offer structured interfaces
to generate log messages, to filter those being important at the moment
(per user configuration), and to present them in a consistent way.

% {structure}
\Tech{gsmon} has three \Keyword[aspects]{structure}.
\Link[\Subject{Messages}]{gsmon:message} enter the subsystem
through the use of \Link[\Code{GSMON_LOG*()}]{GSMON_LOG()} macros.
The \Link[\Subject{monitor}]{gsmon:monitor} receives the messages,
and decides which \Link[\Subject{output driver}]{gsmon:driver}
to send them.  If configured so, the message may not be deliverd
anywhere.  Output drivers are responsoble for formatting the messages
and transmitting them to the display.  \FIGure{gsmon} illustrates the
idea.  Unix administrators may realize similarities to \Prog{syslog-ng}
(which the author has contributed to).

% <CONFIG_GSDIAG_MONITOR>
The logging facility can be disabled at compile time by setting
\Var{CONFIG_GSDIAG_MONITOR} to zero in the library-wide \Link%
[configuration header]{config.h}.  Then no messages will even be
generated, bringing the process to the end at the very beginning,
which may save many CPU cycles if the logging is in a critical
path of the application.

%
% figs/gsmon.tex
%

\begin{NewFig}{gsmon}
\IncludeFig{gsmon}
\caption[The structure of \Tech{gsmon}]%
	{Structure of \Tech{gsmon}.  Differing line styles represent
	sample message routes.	The message of \Subject{Subsystem~1}
	is written to the terminal, while the one of \Subject{Subsystem~2}
	is appended to a file.  The third one is eaten up, because
	the monitor has been configured not to deliver messages of
	\Subject{msgclass~3} at that severity level.}
\end{NewFig}

% End of figs/gsmon.tex


% {messages}, [gsmon:messages]
\Keyword\Target[Messages][messages]{gsmon:message} are actually
composed of several parts beside the actual message text.  The first
of them is the \Subject{message class}.  Callers of \Tech{gsmon}
may group their log messages arbitrarily, for example based on the
activity being performed, such as `init' or `decode'.  The user
may configure the monitor to suppress certain classes of messages,
reducing output noise.  Then, all messages are logged at the specified
\Subject{severity level} which says how important it is: is it intended
for developers, does it warn about a condition that is outght to be
investigated, etc.  The full list of possible severities are documented
in \File{gsmon/constants.h}{src/GSLib/include}.  Finally, the monitor
is passed the location (source file name, method name and line number)
where the message was generated implicitly.

% <GSMON_REGISTER_MSGCLASS()>
% <[GSMON_LOG()]>, <[GSMON_LOGF()]>
% <GSMON_ENTER()>, <GSMON_LEAVE()>
\Topic{logging}
\begin{itemize}
\item	\Fun{GSMON_REGISTER_MSGCLASS}{subsystem, message_class, message_class_id}:\par
	This macro defines a message class for \Tech{gsmon}.  After that
	messages of \Code{message_class_id} can be logged, and will be
	recognized so.  The other parameters are strings, used to display
	the name of the associated entity.
\item	\Target\Fun{GSMON_LOG}{severity, message_class_id, string)}:\par
	Log a string message.  \Code{message_class_id} must already
	have been declared with the previous macro.  The namespace of
	\Code{severity} need not be spelled out.
\item	\Target\Fun{GSMON_LOGF}{severity, message_class, format, }:\par
	Likewise, except that the message text is given \Code{printf()}-%
	style.  This is the way to log numbers, for example.
\item	\Fun{GSMON_ENTER}{}:\par
	Logs at \Code{DEBUG} severity that control has entered a
	function and increases the indentation of further messages.
	This information makes it easier to deduce the dynamic scope
	of a log message.
\item	\Fun{GSMON_LEAVE}{}:\par
	Undoes the effectots of \Code{GSMON_ENTER()}.
\end{itemize}

% {drivers}, [gsmon:driver], <CDriver>, <CDriverStdio>
Output \Keyword\Target[drivers][drivers]{gsmon:driver} are C++ classes
which know how to display a log message.  All of them are derived from
\Class{CDriver}%
%%%%%%%%%
\footnote{All identifiers are in the \Code{gslib::gsmon} namespace.}
%%%%%%%%%
(which is a \Link\Tech{GSObject}).  Currently, only one driver,
\Class{CDriverStdio} is defined, which writes to \Tech{stdio}
\Code{FILE}s, but in theory one could create drivers for any medium.
The drivers are configurable through the public interfaces.

% <setopt_options()>, <setopt_fields()>, <setopt_format()>
% <setopt_time_format()>, <set_tag()>
\begin{itemize}
\item	\Fun{setopt_options}{mask, enable}:\par
	Enable or disable the options specified in \Code{mask}, which is
	a bitmap of \Code{OUTOPT_*} constants.  The constants are defined
	in \File{gsmon/constants.h}{src/GSLib/include}.
\item	\Fun{setopt_fields}{mask, enable}:\par
	Controls what additional information to log along with the message
	text.  \Code{mask} is bitmask of \Code{OUTFLD_*} constants.
\item	\Fun{setopt_format}{mask, enable}:\par
	Controls the appearance of the log messages.  \Code{mask} is
	a bitmask of \Code{OUTFMT_*} constants.  At the moment none
	of the flags is honored.
\item	\Fun{setopt_time_format}{time_format}:\par
	This option is specific to \Code{CDriverStdio}, and sets the
	exact format of the time field in the displayed log message.
\item	\Fun{set_tag}{tag}:\par
	Sets the tag for the \Code{OUTFLD_TAG} field.
\end{itemize}

% {configuration}, [gsmon:monitor]
By default more severe messages (\Code{NOTTICE}\dots\Code{FATAL})
are logged to the standard error.  Classless functions of \Tech{gsmon}
are used to \Keyword\Target[tell][configutation]{gsmon:monitor} the
subsystem how to route messages.

% <set_route()>, <forked()>
\begin{itemize}
\item	\Fun{set_route}{driver, severities, message_classes}:\par
	Instructs \Tech{gsmon} to send messages whose severity is
	one of \Code{severities} and whose message class is one of
	\Code{message_classes} to \Code{driver}.  The ownership of
	\Code{driver} is taken.  If \Code{driver} is \NULL, the route
	is removed.  Returns whether registration was successful.
\item	\Fun{forked}{}:\par
	Output drivers cache the PID of the process for the
	\Code{OUTFLD_PID} field to eliminate frequent calls
	to \Code{getpid(2)}.  When the process \Code{fork()}s,
	its PID changes, and the cache needs to be invalidated.
	This is done by this function.
\end{itemize}
% Logging >>>

\NewSection{Error handling} % <<<

% {[gserr]}, {error stack}
The error handling facility of \Tech{GSLib} is called \Keyword\Target\Tech%
{gserr}.  It is founded around one of the characteristics of errors
that they can be \Subject{chained}.  A failure of a lower layer may
cause failure of its caller, which in turn may make an upper layer
fail etc.  For the user to fully understand a program failure it is
not enough to know the limited information provided by any of the
layers; it is the sum of the information which can frame the context.
The \Keyword\Subject{error stack} of \Tech{gserr} is meant to support
this requirement.  The stack consits of entries describing errors reported
by layers invoked in the latest dynamic path of execution from the least
recent one to the most recent.  The size of the stack is fixed at compile
time, but it can be changed by adjusting \Code{CONFIG_GSDIAG_MAX_ERRORS}
in the global \Link[configuration header]{config.h}.

% <GSERR_PUSH()>, <GSERR_PUSHF()>, <GSERR_PUSH_SYS()>
\begin{itemize}
\item	\Fun{GSERR_PUSH}{message}:\par
	Pushes a new error report onto the top of the error stack.
	\Code{message} is the textual error message, intended for
	humans.
\item	\Fun{GSERR_PUSHF}{format, }:\par
	Likewise, but the message is formatted with \Code{printf(3)}.
	Compare \Link\Code{GSMON_LOG()} and \Link\Code{GSMON_LOGF()}.
	Both this and the previous macro have a \Code{*_EC()} conterpart,
	which allow for supplementing the report with an integer error
	code, which can be used by the receiver to interpret the report
	programmatically.
\item	\Fun{GSERR_PUSH_SYS}{function}:\par
	Like \Code{perror(3)}, but the error message is pushed onto
	the stack.  This macro is intended to report errors returned
	by functions of the C library.
\end{itemize}

% {error propagation}
\Tech{gserr} does not enforce any scheme for the \Keyword[propagation
of the error signal]{error propagation}.  Both C-style return codes
and C++-style exceptions are acceptable.

% <struct error_entry_st>, <GSLIB_THROW*()>, <throwit()>
\begin{itemize}
\item	\Var{struct error_entry_st}:%
	%%%%%%%%%
	\footnote{All identifiers are in the \Code{gslib::gserr} namespace.}
	%%%%%%%%%
	\par
	This structure (defined in \File{gserr/definitions.h}{src/GSLib/%
	include}) holds an error report; in particular, the location where
	the report was generated, the textual error message, and the
	(optional) error code.
\item	\Fun{GSLIB_THROW*}{}:\par
	These are a series of macros defined in \File{gslib/macros.h}%
	{src/GSLib/include} which help you to propagate the error signal
	via exceptions.  Generally, the macros create error reports based
	on the arguments, push them onto the error stack, and throw an
	\Code{error_entry_st} pertaining to this error.  This structure
	can be examined in a \Code{catch} block.
\item	\Fun{throwit}{}:\par
	This function is internal to \Tech{gserr}, but it can be very
	useful during debugging.  Every exception thrown by the macros
	above goes through \Code{throwit()}, exposing a convenient hook
	which can be breakpointed to pause the program when an error is
	reported.
\end{itemize}

% {error examination}
Error reporing is just the half of what is needed for proper error
handling.  The error stack can be \Keyword[examined]{error examination}
by the following functions.

% <lasterr()>, <preverr()>, <forget_all()>, <GSERR_LOG()>
\begin{itemize}
\item	\Fun{lasterr}{}:\par
	Returns the most recent error report from the top of the stack.
\item	\Fun{preverr}{}:\par
	Returns the error preceding \Code{err}.  Together with the
	previous function they can be used to cycle through all the
	reports from the most recent (generated nearest) to the least
	recent (generated deepest in the dynamic scope).
\item	\Fun{forget_all}{}:\par
	Clears the error stack.  Must be called explicitly when the
	error reports are no longer of interest (e.g. the exception
	handler finished processing them).
\item	\Fun{GSERR_LOG}{severity, message_class_id, format, }:\par
	Like \Link\Code{GSMON_LOGF()}, but it also dumps the error stack.
	After that the stack is cleared automatically.
\end{itemize}
% Error handling >>>

\NewSection{Collections} % <<<

% {[GSAvl]}
The collections of \Tech{GSLib} are very versatile data structures.
They try to combine the best breeds of linked lists, hashes, arrays,
stacks, and queues.  Their implementation -- which we refer to as
\Keyword\Target\Tech{GSAvl} -- is based on templates, which makes it
possible to store any types of data in the collections.  \Tech{GSAvl}
complies with the semantics of the \Link<object model> of the library.

% {STL}
The may be raised rightously why not use the Standard Template Library
(\Keyword<STL>), which comes with the C++ compiler for free.  The answer
has just been said: the STL does not (can not) fit in our \Link<object
model>, its classes do not have the expected interfaces and do not
expose expected behavior.  Compensating for these deficiences with
thin wrappers seemed a futile attempt.

% {RAII}
One might further argue that the invention of \Link\Tech{GSObject} was a
waste as a whole, since STL has smart pointers already.  True as it is,
except that \Link\Tech{GSObject} does much more than reference counting.
Besides, the author views the philosophy of the STL (\Term{Resource
Acquisition Is Initialization}, \Keyword{RAII}) a well-marketed hack
to make C++ \emph{appear} as if it had automatic garbage collection.
RAII suggests allocating \emph{all} objects on stack, which \emph{are}
reclaimed when the control leaves the frame.  It is nothing new compared
to C, except that C++ takes the responsibilty to call the destructors of
these objects, so they can clean up after themselves.  This \emph{is}
automatic and garbage collection.  The problem is that dynamically
allocated objects do not enjoy this guardianship, so RAII outlaws them
as legacy heritage.  Of course many things (such as recursive data
structures) cannot be done without resorting to pointers, so more
workarounds (such as the smart pointers mentioned above) have been given
birth to \emph{hide} the problem.  This prolification of kludge is what
the author could not take in.

% [ORIG] "An important aspect of Extreme Programming is not to commit
% to infrastructure before you have to."
Finally, there is the question of extensibility.  Our concern was:
if the collections of STL happen to lack some feature whose neccessity
we cannot decline, is there a way out?  \Link<Extreme programming>
suggests avoiding commitment to a particular infrastructure unless no
other options are feasible, and it does so for good reason: extensible
software is expected to have (relatively) \Link[sane and documented]%
{maintainability} internals, which is often not the case.  To illustrate
the situation of \Prog{libstdc++3}, the then-current version of the
standard C++ library implementation for Linux, let us quote a comment
from a file (\Path{basic_file.h}) distributed with the library:

\begin{quote}
Ulrich \Interposition{the chief \Prog{glibc} supervisor} is going to make
some detailed comment here, explaining all this unpleasantness, providing
detailed performance analysis as to why we have to do all this lame
vtable hacking instead of a sane, function-based approach. This verbiage
will provide a clear and detailed description of the whole object-layout,
vtable-swapping, sordid history of this hack.%
%%%%%%%%%
\footnote{The current version (\Prog{libstdc++6}, \Path{basic_string.h})
has another amusing interposition in it: ``Documentation?  What's that?''.
To be fair, this version seems to be much better commented.}
%%%%%%%%%
\end{quote}

% {structure}, {CTree}, {CNode}
Back to \Tech{GSAvl} \Keyword[details]{structure}, all of its
collections are derivates of the \Keyword\Code{CTree} class%
\footnote{Of the \Code{gslib::gsavl} namespace.}, which holds
together \Keyword\Code{CNode} instances, called \Term{primary nodes}.
The nodes are arranged in an AVL tree, regardless of whether
the actual data structure the collection implements is an array
or a hash.  The code managing the AVL-tree (e.g. rotation) is
an improved version of what was found in \Link\Prog{Aapl}.
The nodes are automatically linked into a list (the \Term{primary list}),
by which they can be enumerated in order.  For this twofoldness,
a \Tech{GSAvl} collection can be regarded either as a list or
as a tree, and it is common to refer to them by these terms.
Primary nodes may head \Term{secondary nodes} (or \Term{siblings})
linked in the \Term{secondary list} (SLL).  The SSL is circular,
for historycal reasons.  Secondary nodes can only be retrieved by
their primary node; list operators will not normally access them.
\FIGure{gsavl} shows the most important relationships.

%
% figs/gsavl.tex
%

\begin{NewFig}{gsavl}
\IncludeFig{gsavl}
\caption[Internal representation of a \Tech{GSAvl} collection]%
	{Internal representation of a \Tech{GSAvl} collection.
	Numbers denote the keys of the nodes.
	Rectangles: ends of the primary list (\Code{list->First()},
	\Code{list->Last()}); hexagons: secondary nodes;
	solid lines: tree edges (bold lines: right subtree)
	(\Code{node->Left()}, \Code{node->Right()});
	dashed lines: secondary list (\Code{node->VNext()});
	dotted lines: primary list (\Code{node->HNext()}).}
\end{NewFig}

% End of figs/gsavl.tex


% {[Array]}
Collections come in three different flavors.  The library user can choose
between them based on what she needs to store.  \Keyword\Target\Tech{Array}s
are best when the elements only have \Term{values}, and they are to be
accessed by index numbers.  \Tech{Array} lists offer operations to add
and remove elements at the head, at the tail, or at any random place,
so it can also be thought of either as a stack, as a queue, or as a
linked list.  One interesting feature of \Tech{GSAvl} \Tech{Array}s
is that indices need not be continuous: there may be an element in the
third and in the nineth slot, but nothing between.

% {[OrSet]}
\Keyword\Target\Tech{OrSet}s (\Term{ordered sets}) are good fit when
the elements shall be looked up by themselves, i.e. when they are the
\Term{keys} to themselves.  This is handy to encode information as the
\Either[presence][absence] of an element in the \Tech{OrSet}.
This flavour of lists does not allow for values.

% {[VOrSet]}
\Keyword\Target\Tech{VOrSet}s are capable of more in exactly this regard.
\Tech{VOrSet}s can be conceived as the \Tech{GSLib} equivalent of Perl
hashes (while \Tech{Array}s coule be modeled as \Tech{VOrSet}s whose
keys of nodes are the indices of the nodes in the list).  All three
flavors support the ordered enumeration of their elements (this is why
\Tech{OrSet}s are `ordered').

% {VaList}
Two other types of lists are built upon these flavors.  They are
jointly referred to as \Keyword\Tech{VaList}s, and the speciality
about them is that they can do math with their values.  For example,
it is possible to numerically add matching elements of two lists,
or to multiply all values of a list by a scalar.  \Code{TVaList}
show the characteristics of an \Tech{Array}, while \Code{TKaList}
is like a \Tech{VOrSet}.

% {declaring collections}
As suggested earlier, the exact data type of keys and values
does not matter to \Tech{GSAvl}, apart from that keys must be
\Link\Code{IComparable}.  Keys and values are embedded in
\Link<containers> in the nodes.  This makes the concrete type
of nodes and lists depend on the type of the keys and values,
therefore the appropriate template classes must be \Keyword%
[instantiated]{declaring collections} to enter into the posession
of a collection.  The macros of \File{GSAvl/macros.h}{src/GSLib/include}
help in this task.  See the header for the full range of possibilities.
See \File{test/simple.cc}{src/GSLib} for examples.

% [GSAVL_*()], <GSAVL_ARRAY()>, <GSAVL_ORSET()>, <GSAVL_VORSET()>
\Topic<GSAVL_*()>
\begin{itemize}
\item	\Fun{GSAVL_ARRAY}{value_kind, value_type}
\item	\Fun{GSAVL_ORSET}{key_kind, key_type}
\item	\Fun{GSAVL_VORSET}{(key_kind, key_type), (value_kind, value_type)}:\par
	These macros expand to the C++ type of a list, which can be
	\Code{typedef}ed or used to define a variable.  The semantics
	of the arguments follow that of \Link\Code{GSOBJ_CONTAINER()}.
	Note the parentheses in the arguments of \Code{GSAVL_VORSET()};
	they are \emph{required}.
\end{itemize}

% {[list-wide operations]}
The heart of \Tech{GSAvl} collections lie in the operations one can
do about them.  The \Keyword\Target[first group of operations]{list-wide
operations} affect the list as a whole.  Since lists are full-rank
\Link\Tech{GSObject}s, the `standard' interfaces are implicitly available
as well.

\begin{itemize}
% <configure()>
\item	\Fun{configure}{parameter, value}:\par
	With this method you can configure the run-time behavior of
	the list with respect to unanticipated events, such as when
	a particular node is searched but none found.  The defaults
	are biased toward the C-style `check the return' scheme, but
	it may not always be the best choice.
	See \File{GSAvl/definitions/CTree.h}{src/GSLib/include}.

% <size()>, <volume()>
\item	\Fun{size}{}:\par
	Returns the number of primary nodes in the list.
\item	\Fun{volume}{}:\par
	Returns the number of all nodes (primary $+$ secondary) in the list.

% <trim()>, <truncate()>
\item	\Fun{trim}{}:\par
	Eliminate all secondary nodes from the list.
\item	\Fun{truncate}{}:\par
	Empty the list completely.

% <merge()>, <subtract()>, <intersect()>
\item	\Fun{merge}{other, depth}:\par
	Add all nodes of the \Code{other} list to the receiver.  In case
	of \Tech{Array}s \Code{other} is effectively appended to the
	receiver.  For other flavors of lists the result depend on the
	list configuration; by default nodes whose key does not exist in
	the receiver are added, otherwise they replace their conterparts.
	Unless directed otherwise nodes are deep-copied.
\item	\Fun{subtract}{other}:\par
	Remove all nodes whose keys are present in the \Code{other} list.
	This operation is not defined for \Tech{Array}s.  (Rationale:
	in \Tech{Array}s there are not keys.)
\item	\Fun{intersect}{other}:\par
	Only keep those nodes whose keys are present in the \Code{other}
	list.  This operation is not defined for \Tech{Array}s.
\end{itemize}

% {list node operations}
The second group of operations are which deal with \Keyword[individual
nodes]{list node operations}, and they are many and numerous.  Library
users will spend most time with these methods, and may recognize heavy
Perl influence.  Almost all of the methods take multiple calling
conventions; these are indicated in the suffix of the method name.
For example, one version of \Code{find()} returns the node found, another
returns with its value directly.  Here we only list the versions taking
or returning nodes; the semantics of the others are hopefully deducible
from these descriptions examining their prototypes.

\begin{itemize}
% <node_t>, <mknode0()>, <is_in_list()>, <idx_n()>
\item	\Var{node_t}:\par
	The exact C++ type of nodes depend on the type of their keys
	and values.  It is possible to recreate the node types by
	the \Link[\Code{GSAVL_*_NODE()}]{GSAVL_*()} macros of
	\File{GSAvl/macros.h}{src/GSLib/include}, but the \Code{node_t}
	member of list classes has exactly the same meaning.
\item	\Fun{mknode0}{}:\par
	Returns a newly allocated node of the appropriate type
	(\Code{node_t}), but does not add to the list.  The key
	\AndOr the value of the node are \Link[null]{null value}.
\item	\Fun{is_in_list}{node}:\par
	Returns whether \Code{node} is in the list, either as a primary
	or a secondary node.
\item	\Fun{idx_n}{node}:\par
	Returns the number of nodes ranked less than \Code{node} in the
	primary list.  Indexing starts from zero.

% <unshift_n()>, <push_n()>, <pred_n()>, <succ_n()>, <enter_n()>
\item	\Fun{unshift_n}{node}:\par
	Inserts \Code{node} at the beginning of the list.
	Only meaningful in \Tech{Array} context.%
	%%%%%%%%%
	\footnote{In theory this and the following few methods can be
	used in \Tech{OrSet} context as well, given that the key of
	\Code{node} compares between \Code{where} and its predecessor,
	otherwise the integrity of the list is ruined.}
	%%%%%%%%%
\item	\Fun{push_n}{node}:\par
	Inserts \Code{node} at the end of the list.
	Only meaningful in \Tech{Array} context.
\item	\Fun{pred_nn}{where, node}:\par
	Inserts \Code{node} just preceeding \Code{where}.
	(Think \Code{splice()} of Perl.)
	Equivalent to \Code{unshift_n()} if \Code{where} is \NULL.
	Only meaningful in \Tech{Array} context.
\item	\Fun{succ_nn}{where, node}:\par
	Inserts \Code{node} just succeeding \Code{where}.
	Equivalent to \Code{push_n()} if \Code{where} is \NULL.
	Only meaningful in \Tech{Array} context.
\item	\Fun{enter_n}{node, depth}:\par
	Adds \Code{node} to the list at the appropriate place.
	For \Tech{Array}s it is equivalent to \Code{push_n()}.
	\Code{depth} observes \Link\Code{copy()} semantics, i.e.
	ownership of \Code{node} is taken if \Code{depth} is zero,
	ownership is shared if the parameter is one, otherwise
	a \Link\Code{dup()} of \Code{node} is added to the list.

% <nth_n()>, <any_n()>, <[find_n()]>
\item	\Fun{nth_n}{index}:\par
	Returns the node at the \Code{index}th position in the primary
	list of the collection.  Think of it as a subscript operator.
	If there is no such node the result depends on how the list is
	\Code{configure()}d.
\item	\Fun{any_n}{}:\par
	Returns a \Code{rand(3)}omly chosen node from the list.
\item	\Target\Fun{find_n}{key}:\par
	Finds and returns the node whose key is \Code{key}.
	Does not search the secondary list of nodes.
	Only meaningful in \Tech{OrSet} and \Tech{VOrSet} contexts.

% <[unlink_n()]>, <destroy_n()>, <shift_n()>, <pop_n()>
\item	\Target\Fun{unlink_n}{node, is_chainwide}:\par
	Removes \Code{node} from the list.  Returns \Code{node},
	its ownership given away.  If the operation \Code{is_chainwide}
	\Code{node} will keep its siblings (which then cease to be part
	of the list); otherwise the first sibling of \Code{node} takes
	its place.
\item	\Fun{destroy_n}{node, is_chainwide}:\par
	Removes \Code{node} from the list and \Link\Code{release()}s it.
	If the operation \Code{is_chainwide} siblings of \Code{node}
	will go away too, otherwise the first one takes its place.
\item	\Fun{shift_n}{}:\par
	Unlinks and returns the first node of the list.
\item	\Fun{pop_n}{}:\par
	Unlinks and returns the last node of the list.
\end{itemize}

% {standalone node operations}
Nodes are not necessarity part of any list; they can be created
standalone to enter a list later, or be detached from one via
\Link\Code{unlink_n()}.  The following \Keyword[operations]%
{standalone node operations} are available in either case.

\begin{itemize}
% <key>, <value>
\item	\Var{key}, \Var{value}:\par
	These fields embed the key \AndOr value of the node.
	They are \Link<containers>, so to access the value of
	a node for example one writes \Code{node->value.payload}
	or \Code{node->value.retrieve()}.

% <is_primary()>, <get_head()>
\item	\Fun{is_primary}{}:\par
	Tells whether the node is part of a list and is primary,
	i.e. \Link\Code{find_n()} can find it.
\item	\Fun{get_head}{}:\par
	Returns the primary node in the secondary list of the node.
	If the receiver \Code{is_primary()} then it returns itself.

% <is_in_list()>, <get_list()>
\item	\Fun{is_in_list}{list}:\par
	Tells whether the receiver is in \Code{list}, or in any list
	if the argument is \NULL.
\item	\Fun{get_list}{}:\par
	Returns the object reference to the list the node is part of,
	or \NULL if there is no such list.

% <HPrev()>, <HNext()>, <VPrev()>, <VNext()>
\item	\Fun{HPrev}{}, \Fun{HNext}{}:\par
	Return the node \Either[preceeding][succeeding] the receiver
	in the primary list.  Standalone and secondary nodes are not
	on primary lists, hence the methods return \NULL for them.
\item	\Fun{VPrev}{}, \Fun{VNext}{}:\par
	Return the \Either[previous][next] node in the secondary list
	of the receiver.  The secondary list is circular at both ends.

% <has_sibling()>, <make_subling()>, <unmake_sibling()>, <kill_siblings()>
\item	\Fun{has_sibling}{}:\par
	Tells whether the node has at least one sibling, i.e.
	\Code{VNext()} would not return the receiver itself.
\item	\Fun{make_sibling}{other}:\par
	Joins the secondary lists of the receiver and the \Code{other}
	node such that the former is broken open and pasted between
	\Code{other} and its left (\Code{VPrev()}) sibling.
	\File<src/GSLib/CNode.cc> has a nice diagram showing
	the exact layout.
\item	\Fun{unmake_sibling}{}:\par
	Unlinks the receiver from its secondary list.  The node must
	not be primary; use the \Link\Code{unlink_n()} list operation
	in that case.
\item	\Fun{kill_siblings}{}:\par
	\Link\Code{release()}s all the siblings of the receiver, which
	must either be primary or not part of any list.

% <dismiss()>
\item	\Fun{dismiss}{}:\par
	\Link\Code{release()}s the receiver node, and returns its value.
\end{itemize}
% Collections >>>

\NewSection{XML driver} % <<<

% {[GSXml]}
\Topic{overview}
The purpose of the XML driver of \Tech{GSLib} is to allow developers to
create XML printouts quickly and easily.  To achieve that the first task
is to build the internal \Keyword\Target\Tech{GSXml} representation of
the XML document.  From the library user's perpective this usually is
done by invoking the \Link\Code{serialize()} standard \Link\Tech{GSObject}
interface, which returns a \Tech{GSXml} object representing an XML subtree:
the receiver and other objects that logically belong to it.  It is possible
to traverse and manipulate this tree; however \Tech{GSXml} is optimized
to extending and finally printing the XML structure to a file stream.
This is realized by calling the \Link\Code{print()} standard interface
of the returned \Tech{GSXml} object.

% <print_it()>, <[start_output()]>, <end_output()>
The carefree \Fun{print_it}{} function is provided to automate the whole
procedure.  It has two accompanying functions, \Target\Fun{start_output}{}
and \Fun{end_output}{}, which ensure that the output is a valid XML document.

% {structure}
The \Tech{GSXml} \Keyword[representation]{structure} is a very simple,
tree-like structure, which describes what XML nodes constitute the
document and how they are organized.  \FIGure{gsxml} gives an idea about
the high-level structure of the tree.  From the driver's point of view
an XML document consists of nodes (\Code{<Something></Something>}),
node attributes (\Code{<Something attr="value"/>}), and freestanding text.
They are mapped to \Code{CNode}%
%%%%%%%%%
\footnote{Not to be confused it with \Code{gslib::gsavl::CNode}.
All identifiers introduced in this section are located in the
\Code{gslib::gsxml} namespace.},
%%%%%%%%%
\Code{CAttr}, and \Code{CLeaf} class of objects respectively, all of
them being proper \Link\Tech{GSObject}s, and proper \Link\Tech{GSAvl}
nodes at the same time.

%
% figs/gsxml.tex
%

\begin{NewFig}{gsxml}
\IncludeFig{gsxml}

\begin{lstlisting}[style=XML,basicstyle=\small\ttfamily]
<Article type="editorial" year="1982" reviewed="true">
	<Authors>
		<Author initials="L." last_name="Joseph"/>
		<AndOthers/>
	</Authors>
	<Title>My apologies</Title>
	When I was born...
</Article>
\end{lstlisting}

\caption[Internal \Tech{GSXml} representation of an XML document]%
	{The internal \Tech{GSXml} representation of the XML document
	underneath.  (Read left-to-right, top-to-bottom.)  Framed
	drawings symbolize objects of the corresponding class set at
	the top of the drawings.  Below that are the relevant fields
	of the objects.  Arrows denote \emph{has} relationship between
	objects.}
\end{NewFig}

% End of figs/gsxml.tex


% {CNode}
As it should be clear from the figure, the tree is defined recursively.
In essence, a \Keyword\Code{CNode} is a handle for a (sub)tree.
A \Tech{GSXml} node may have attributes, may contain freestanding text,
and may link to other nodes.

% <tag>, <attrs>, <children>, <find_child_by_tag()>, <ctor()>
\begin{itemize}
\item	\Var{tag}:\par
	This is the strings what \Link\Code{print()}s between the
	angle brackets.  It is not \Code{free()}d upon the destruction
	of the node, so it is advisable not to assign it a dynamically
	allocated string.
\item	\Var{attrs}:\par
	This is a \Link\Tech{OrSet} of \Code{CAttr}s, and it can be
	manipulated by regular \Link<list-wide operations>.  The set
	is indexed by textual attribute names.
\item	\Var{children}:\par
	This is an \Link\Tech{Array}-like collection, gathering
	both \Code{CNode}s and \Code{CLeaf}s mixed in any order.
	The children are \Link\Code{print()}ed in the order they
	are in \Code{children}.
\item	\Fun{find_child_by_tag}{tag}:\par
	Would be \Code{children.find_n(tag)} if it were indexed.
	But it is not, so this methods searches \Code{children}
	linearly, and returns the first node whose \Code{tag} matches.
\item	\Fun{ctor}{tag}:\par
	This constructor initializes \Code{tag} to the specified value.
	Since \Code{tag} must always be set, this is the only wasy to
	create a \Code{CNode} object from the scratch.
\end{itemize}

% {CAttr}
While \Tech{GSXml} attributes are collected in a pure \Link\Tech{OrSet},
\Keyword\Code{CAttr}s \emph{do} have a value---it is only that this fact
is concealed from its list to allow for varatic attribute value types.

% <value>, <ctor()>
\begin{itemize}
\item	\Var{value}:\par
	This is a \Link[varadic container]{CContainerVaradic}
	of the attribute value.
\item	\Fun{ctor}{name, value}:\par
	Creates a \Code{CAttr} object, initializes its key
	and value.  \Code{value} can be of any C++ data type
	\Link\Code{CContainerVaradic} is ready to handle.
\end{itemize}

% {CLeaf}
\Keyword\Code{CLeaf}s, representing freestanding text content in an
XML node, are much like \Code{CAttr}s, except that they do not have
an indexing key.  We note here that textual content only undergo basic
encoding when \Link\Code{print()}ing it, i.e. the characters `<', '>',
'\&', and '"' are encoded, but nothing else is (such as 8-bit accented
characters).  The \Code{charset} argument of \Link\Code{start_output()}
may help to preserve the XML-validity of the output.

% {protocol}
Developers who wish their object to be serialisable need to
\Keyword[implement]{protocol} the \Link\Code{ISerialisable}
interface, which consist of just one method:

% <serialize()>
\begin{itemize}
\item	\Fun{serialize}{options, base}:\par
	This method returns a \Tech{GSXml} \Code{CNode}, suitable to be
	added to the \Code{children} of another node.  If \Code{base} is
	not \NULL all contents must be added to that node; this is for
	the caller in order to control where its objects serialize
	themselves.  Otherwise the base node needs to be created,
	setting its \Code{tag} appropriately.  Then add \Tech{GSXml}
	attributes and children nodes as best represent the object,
	possibly by asking subordinate objects to \Code{serialize()}
	themselves, and adding the returned nodes to the \Code{children}
	of the \Code{base} node.  \Code{options} is a \Link\Tech{VOrSet}
	of strings pointing to unsigned integers, which is intended as a
	platform free from all encumbrances for communicating directives
	about what the caller wants to be included in the dump.  Beware
	that \Code{options} can be \NULL. When serialization is done,
	return the base node.
\end{itemize}
% XML driver >>>

\NewSection[Test suite]{gslib:test} % <<<

\Topic{overview}
Basically, the \Link<second generation> \Tech{GSLib} was raised over its
predecessor for the sole demand of \Link<collections>.  It may not be
surprising then that the test suite of the library has \Link\Tech{GSAvl}
in focus. but in such a way that the other subsystems are tried just as
well with good coverage.  The suite is a compilation of standalone test
programs, most of them aiming at \Link[streessing]{stress testing} the
implementation.  The test programs also serve as source of examples on
how to use a particular function of the library.  The \Link<build system>
has \Prog{make} targets to run tests, with multiple configurations.
It is worth noting that the \File{Makefile}{src/GSLib/test} controls
what debugging libraries (such as \Link\Prog{efence} or \Link\Prog{ccmalloc})
shall be linked with the test programs.  As noted at the discussion of
the \Link<developer tools> these libraries carry a performance penalty,
which can be unacceptable in long-ranging test runs.

% [SKIP] valists
\begin{itemize}
\item	\File[simple]{src/GSLib/test/simple.cc}:
	The purpose of this simple test program is to confirm \Tech{GSLib}
	is functional (i.e. did not break in a fundamental way due
	to a recent change).  The test is successful if the program
	compiles, links, runs, and prints an XML dump of a list.
	Looking at the source one can get insight into the basic usage
	of \Link\Tech{GSObject}, \Link\Tech{GSAvl}, \Link\Tech{GSXml},
	and \Link\Tech{gserr}.
\item	\File[memory]{src/GSLib/test/memory.cc}:
	Allocates a \Link[collection]{collections} and fills it with the
	specified (large) number of items, displaying statistics about
	the memory consumption regularly.  The use of this program is to
	learn how well \Link\Tech{GSAvl} scales, with references to memory
	and processor resources.
\item	\File[trees]{src/GSLib/test/trees.cc}:
	This is the main test for \Link\Tech{GSAvl}.  It has two modes of
	operation: Running on its own it fills a collection with random
	items, then removes them in the same order.  This is not very
	interesting, but in the early days of development it allowed us
	to examine the AVL properties of the trees as they were growing.

	In the other mode it's playing as the mate of \File{trees-co.pl}%
	{src/GSLib/test} in a \Link<co-testing> intercourse.  This case
	the Perl program commands the mate what to do about the list,
	both parties perform the operation, then compare the contents
	of their lists.  The exact sequence of \Either[commands][events]
	can be reproduced and recorded, greatly adding to debug information.

	The \Link<build system> knows about the following test configurations:

	% <make treetest>, <make qtreetest>,
	% <make cotreetest>, <make qcotreetest>,
	% <make hcotreetest>
	\begin{itemize}
	\item	\Make{treetest}:
		Test \Link\Tech{Array}s and \Link\Tech{VOrSet}s thoroughly
		standalone.  Can take hours if \Link\Code{CONFIG_DEBUG}
		is \Code{full}.
	\item	\Make{qtreetest}:
		Likewise, but make it shorter.
	\item	\Make{cotreetest}:
		Launch a long test in \Link<co-testing> mode.
	\item	\Make{qcotreetest}:
		Likewise, but keep it short.
	\item	\Make{hcotreetest}:
		Run a series of very long \Code{cotreetest}s.
		We found it useful in isolating deeply hidden
		memory leaks.
	\end{itemize}
\item	\File[compare]{src/GSLib/test/compare.cc}:
	This is the \Link<co-testing> mate of \File{compare-co.pl}%
	{src/GSLib/test}.  Together they examine the \Link\Code{compare()}
	interface of \Link\Tech{GSAvl} lists by the comparison of random
	collections.

	% <make acmptest>, <make scmptest>, <make cmptest>
	\begin{itemize}
	\item	\Make{acmptest}:
		Test \Link\Tech{Array}s.
	\item	\Make{scmptest}:
		Test \Link\Tech{OrSet}s.
	\item	\Make{cmptest}:
		Test one after the other.
	\end{itemize}
\item	\File[buckets]{src/GSLib/test/buckets.cc}:
	This program demonstrates the use and tests the implementation
	of \Link<containers> and buckets.  (\Term{Buckets} have a
	structured interface for throwing objects at, and they are
	supposed to select one of them by some undefined criteria.)
\end{itemize}
% Test suite >>>

%%%%%%%%%%%%%%%%%%%%%%%%%%%%%%%%%%%%%%%%%%%%%%%%%%%%%%%%%%%%%%%%%%%%%%
\NewChapter{In closing}

% <<<
We have overviewed the functional structure of \Menugene, and examined
the basic theory of genetic algorithms as applied to the problem of
optimal dietary menu generation.  Later on we gained insight into the
development environment---the workplace of the developers.  Lastly we
went through the internals of \Link<GSLib>, which constitutes large
part of the framework that the problem solver engine relies on, and is
arguably the most difficult to understand amongst the system components.

Acknowledging the fact that \Menugene stands on its own, and can generate
menus fulfilling its \Link<purpose> we can conclude that genetic algorithms
are a feasible method for this kind of problems.  It has also shown great
potentials, which, however, needs considerable \Link<future> research.

From the software development process we have learnt a number of lessons.
First, \Link[rewriting software from scratch]{avl} is the luxury of those
who possess unrestricted amount of resources.  (\cite{Joel} characterizes
it as ``the \textbf{single} worst strategic mistake that any software
company can make''.) It indeed can pay off very well, but the iterative
development approach as advocated by \Link<extreme programming> provides
much safer grounds for business software manufacturing.

The appropriate choice of programming tools and languages may compensate
for these time delays.  In our case \Prog{Perl} offered an appealing
alternative to C++, even in areas (e.g. GUI programming) that are not
commonly thought of as the prime examples of the applicability of script
languages.  Its introduction meant significant boost to the project,
visible in the rapid growth in the number of applications, such as
\Link\Tech{rulez} and \Link\Tech{subborn}.

Finally, we have found that it is worth investing time into the development
tools.  Our \Link\Tech{BITS} served us very well in case of trouble, which
came inevitably.  They have an important role from another aspect as well:
these are the homegrown pet programs that have the best chance for living
on then the current project is finally closed.
% In closing >>>

% vim: set comments=O\:% commentstring=%\ %s:
% vim: set foldmarker=<<<,>>> foldmethod=marker:
% vim: set hlsearch ignorecase smartcase formatoptions+=or:

%
% figs/fitness.tex
%
\begin{NewFig}{fitness}
\IncludeFig{fitness}

\def\cint#1{\MathVar{cint.#1}}
\def\odiff#1{(\cint{o} - \cint{#1})^2}

\small

\[
	\MathFun{fitness}(\MathVar{cset}) =
		\sum_{\MathVar{cint} \in \MathVar{cset}}
		\min(0, \MathFun{fit}(\MathVar{cint}))
\]

\begin{displaymath}
\hfuzz2.2pt% no, it doesn't overfill
\setlength\arraycolsep{2pt}
\MathFun{fit}(\MathVar{cint}) =
\left\{
\begin{array}{l|lcl}
-\odiff{v} + \odiff{l} - 1
	& \cint{v} & \le & \ \cint{l}		\\
-\frac{\odiff{v}}{\odiff{l}}
	& \cint{v} & \in & ( \cint{l}; \cint{o})\\
0
	& \cint{v} &  =  & \ \cint{o}		\\
-\frac{\odiff{v}}{\odiff{r}}
	& \cint{v} & \in & ( \cint{o}; \cint{r})\\
-\odiff{v} + \odiff{r} - 1
	& \cint{v} & \ge & \ \cint{r}		\\
\end{array}
\right.
\end{displaymath}

\caption[The fitness function of \Menugene]%
	{The fitness function of \Menugene. \cite[pp.~61--64]{TDK}
	\MathVar{cset} is a set of constraints (\MathVar{cint}).
	\cint{l}, \cint{o}, and \cint{r} are parameters of the curve
	(the accepted minimum, the optimal, and the tolerated maximum
	amount of a nutrition component in the solution being measured),
	while \cint{v} is the actual amount of the corresponding component.
	The graph depicts an example one-dimensional curve parametrized
	with $\{ 1, 3, 7 \}$.  This function, in fact, \emph{penalizes}
	deviations from the optimum. (Our implementation also negates the
	fitness score at the end, but it does not invalidate the point.)
	At the bounds (pointed to by arrows) the penalty jumps very
	high, making it infeasible that the solution will compare well
	with other solutions.}
\end{NewFig}

% End of figs/fitness.tex

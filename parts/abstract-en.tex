%
% parts/abstract-en.tex -- English abstract
%

\begingroup
\vspace*{1cm}\vspace*{1ex}% ???
{\centering\large\bfseries\MakeUppercase{Abstract}\par}
\vspace*{12ex}

\itshape\footnotesize

This thesis is the technical and historical account of the \Menugene
project, conducted by the Department of Information Systems and developed
in joint partnership with the Department of Dietetics of the Semmelweis
University, Budapest, which the author has participated in as a software
developer.  The project goal is to establish a computer service for the
layperson to help in the compilation of personal dietary menus that offer
optimal nourishment and satisfy the consumer's taste as well.

The system is composed of several layers and components: standalone
user interfaces and a web interface for user interaction, a database
for static nutrient data storage, and a system service that solves
the problem of menu generation per user request, all linked through
network.  The problem solver logic is based on genetic algorithms.

Mature or experimental technologies exist for all of these components.
The completed objective of the project was to fit them together in
appropriate implementations, and to create a convenient research platform
for the further improvement of the problem solver engine, taking special
note of the future maintainability of the product.

To this end several programs, applications, libraries and developer
utilities have been written in Perl, C, C++ and PHP languages, and
the inherited database schema has been reworked.  The source code
is equipped with excessive documentation, which the present thesis
is also part of.  As to the current state of affairs, the system is
demonstratable.

This document covers the software engineering aspects of \Menugene.
It reviews the tools and practices that characterized the development
process, along with the implementation technologies and methods, aiming
at giving sufficient explanations concerning decisions for or against
a particular item.  A detailed overview of genetic algorithms is also
included.

\bigskip
\upshape\bfseries
\noindent
Keywords: \Menugene, nourishment, genetic algorithms, multi-objective
optimization, software engineering, defensive programming
\endgroup

% End of parts/abstract-en.tex

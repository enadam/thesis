%
% parts/guide.tex -- some cheer up chatter
%

\clearpage
\section*{How to read this document}
\markright{\MakeUppercase{How to read this document}}
\thispagestyle{plain}

% /intended audience/
This text \Target[aims]{docaim} at documenting the development process
of \Menugene as perceived by the author, in the inarticulate hope of
attracting successors who would carry on the work.  In this regard it
can be read as a manual, which helps in finding the way around while
making improvements to the system.  Those who have this intention will
definitely learn most from the \Link[third chapter]{GSLib}.  The ones
who do not wish to dwell into technical details, but are interested in
the \emph{what}s are best served by the \Link[first chapter]{executive
summary}.  Finally, those professionals who are not very excited by
the high-level mission of the system, nor by its implementation nuts and
bolts may find the \Link[second chapter]{overview of the implementation}
a good reading into software engineering.

% /scope/
The scope of this document is limited to those bodies of knowledge that
are not apparent from the source code.  This is to avoid overlaps, which
would inevitably reduce the usefulness of this text.  It follows that
we abstained from including lengthy code samples.  For this the source
codebase should be consulted, which we tried to make as seamless as
possible by the means of crossreferences.

% {annotation}
The text is extensively \Keyword[annotated]{annotation} and
crossreferenced to foster the rapid overview of sections and the
quick localization of additional information.  In the electronic
version nearly all paragraphs are bookmarked, which gives the
impression of a very detailed table of contents if displayed
with a suitable viewer.  This case the crossreferences behave
as hyperlinks.  In the hardcopy edition this is emulated by
indicating the referred page number on the margin, out of the
text body to minimize distraction in reading.

% /colors/
\makeatletter
Crossreferences are colored differently, depending on the type of
their destinations.  \textcolor{\@linkcolor}{Internal links} take the
reader to another point within the text body, including tables and
figures.  \textcolor{\@citecolor}{Citations} point to entries in the
\Link<bibliography> or in the \Link<progliography> (list of referred
software).  \textcolor{\@urlcolor}{External references} are meant to
help in looking up the authentic source of information, which can be
checked up.  Unless it is clearly visible otherwise these links point to
files and directories in the source codebase.  To leverage the immediate
access offered by them the repository must be extracted in \Path{/src}.
\makeatother

% /style conventions/
Many words in the text are set in distinctive typefaces to facilitate
the semantic decoding of dense technical material.  Code fragments
and path elements are set in \Code{typewriter} style, allowing for
the unambiguous identification of functions and variables.  Proper
technical names are \Tech{sans serif}, while \textsc{small caps} are
reserved for concrete, widely recognized software names cited from
the \Link<progliography>.  \emph{Italics} emphasize key points in
a phrase, or markup auxiliary subjects.

% [dbend]
\Warning
Occasionally a \Target[little sign]{dbend}\footnote{Borrowed from
\cite{TeXBook}, whose author placed it into public domain.} demands
attention on the margin, warning that the following topic might be
hard to swallow.  Its rough interpretation is that we suggest that
readers who cannot make sense of the paragraph in question at the
first sight just ignore the explanations and accept the statements
as they are.

% {colophon}
This document was \Keyword[typeset]{colophon} by \Prog{TeX},
employing hundreds of custom macros organized into an integrated
package developed by the author while writing this thesis.  Most
figures have been prepared with \Prog{Graphviz}, its \Tech{PostScript}
output being heavily customized by other homegrown tools.  In this
work \cite{PostScript1,PostScript2} have proven indispensable.
During composition the safety and the consistency of the document
sources were guarded by the \Prog{darcs} revision control system,
which turned out to be extremely suited for this kind of situations.

% End of parts/guide.tex
